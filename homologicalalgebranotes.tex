\documentclass[12pt]{article}
\usepackage{graphicx}
\usepackage{amsmath}
\usepackage{fontawesome5}
\usepackage{booktabs}
\usepackage{amssymb}
\usepackage{amsthm}
\usepackage{lmodern}
\usepackage[english]{babel}
\usepackage[utf8x]{inputenc}
\usepackage[toc,page]{appendix}
\usepackage[nottoc]{tocbibind}
\numberwithin{equation}{section}
\graphicspath{ {./Images/} }
\usepackage[raggedright]{titlesec}
\usepackage{placeins}
\usepackage{tikz}
\usepackage{mathtools}
\usepackage{float}
\usepackage[autostyle]{csquotes}\usepackage{quiver}
\usepackage[activate={true,nocompatibility},final,tracking=true,kerning=true,spacing=true,factor=1100,stretch=10,shrink=10]{microtype}
%\usepackage{hyperref}

\newcommand{\R}{\mathbb{R}}
\newcommand{\Q}{\mathbb{Q}}
\newcommand{\C}{\mathbb{C}}
\newcommand{\Z}{\mathbb{Z}}
\newcommand{\N}{\mathbb{N}}
\newcommand{\F}{\mathbb{F}}
\newcommand{\Hom}{{\mathrm{Hom}}}
\newcommand{\image}{{\mathrm{Im}}}
\newcommand{\kernel}{{\mathrm{Ker}}}
\newtheorem{theorem}{Theorem}[section]
\newtheorem{definition}{Definition}[section]
\newtheorem{corollary}{Corollary}[theorem]
\newtheorem{lemma}[theorem]{Lemma}

\newtheorem{proposition}{Proposition}[section]
%opening
\title{Sketches on homology}
\author{Bhoris Dhanjal}

\begin{document}
	\tableofcontents
	\maketitle
	The material is presented mostly module theoretically with appropriate categorical abstractions when appropriate (and pedagogically relevant) to better elucidate the concepts in play. We give a proof sketch of the Freyd-Mitchell embedding at the end and demonstrate usage of abelian categories. The overall structure of the presented material is influenced by the appendix on homological algebra in Eisenbud's excellent book \cite{eisenbud2013commutative}.

	\section{Modules and tensor products}
	\begin{definition}[Module]
			Recall that for a commutative ring with unity $R$ a \textbf{module} over $R$ is a set $M$ along with operations of addition $+:M\times M \to M$ and scalar multiplication $R\times M \to M$ such that 
		\begin{itemize}
			\item $(M,+)$ is an abelian group,
			\item Scalar multiplication distributes over addition,
			\item Scalar multiplication respects unity and associativity.
		\end{itemize}
	\end{definition}
	
	\begin{definition}[Bilinear maps between modules]
		It is a map in two variable map which is linear over its arguments individually (when the other is held fixed).
	\end{definition}
	
	\begin{definition}[Tensor product]
			The \textbf{tensor product} for $R-$ modules $M,N$ denoted as $M \otimes_R N$ \footnote{When the choice of ring is obvious we will drop the subscript.} is an $R$ module along with a universal $A-$bilinear map $\varphi:	M \times N \to M \otimes N$. That is to say all $R-$bilinear maps factor through the tensor product, this is elucidated by saying the following diagram commutes for all $ R-$modules $P$ and $f$ some bilinear map from $M \times N$ into $P$
		\[\begin{tikzcd}
			{M\times N} && {M\otimes N} \\
			\\
			&& P
			\arrow["\varphi", from=1-1, to=1-3]
			\arrow["f"', from=1-1, to=3-3]
			\arrow["{\exists! \tilde{f}}", dashed, from=1-3, to=3-3]
		\end{tikzcd}\]
	\end{definition}
	

	
	In order to appreciate this universality first lets see the explicit construction of the tensor product.
	\subsection{Explicit construction of tensor products}
	The universality is the key point via which we aim to define a explicit construction. With it in mind the typical construction goes as such, consider the free module generated by the basis set $M \times N$. 
	
	Quotienting out the ideal which is generated by the bilinearity conditions we require will give us the explicit module we require. The required bilinearity conditions are,
	\begin{itemize}
		\item $(m+m', n)-(m,n)-(m',n')$ and $(m,n+n')-(m,n)-(m,n')$
		\item $(xm,n)-x(m,n)$ and $(m,xn)-x(m,n)$
	\end{itemize}
	For all $m,m'\in M, n,n' \in N, x \in R$. It should be clear from the definitions that identifying these points together makes the quotient bilinear as required and it is the smallest such module as there we have added no extraneous conditions.	This is easy to verify any maps $f: M\times N \to P$ as above will extend naturally to $F: R^{\otimes M\times N} \to P$ and is factoring through the tensor product mapping say $\varphi: M\times N \to M \otimes N$
	
	\textbf{COMPLETE THIS PROOF
	CITE THIS SOMEWHERE}
	
	\subsection{Properties of tensor products}
	Let $R$ be some commutative ring with unity and let $M,N,P$ be $R-$modules.
	\begin{proposition}[Commutativity]
		$M \otimes N = N\otimes M$
	\end{proposition}
	\begin{proof}
		Note that the map $ (m,n)\to(m,n)$ induces a isomorphism between the free groups generated by the products. The conditions for bilinearity aren't affected so the quotients are the same.
	\end{proof}
	\begin{proposition}[Identity]
		$R \otimes M = M$
	\end{proposition}
	\begin{proof}
		Consider the canonical bilinear map $f: R \times M \to P$ given by $(r,m)\mapsto rm$. $M$ satisfies the UMP for this.
	\end{proof}
	\begin{proposition}[Associativity]
		$(M \otimes N) \otimes X = M \otimes (N \otimes X)$	
	\end{proposition}
	\begin{proof}
		Direct consequence of the ump simply draw the diagrams first starting with $M \times N \to P$ and then $N \times X \to P$
	\end{proof}
	

	
	\subsection{Tensor-Internal Hom adjunction}
	As covered in the appendix we have a notion of an internal hom functor as a bifunctor $[-,-]:  \mathbf{C}^\mathrm{op} \times \mathbf{C} \to \mathbf{C}$ for some symmetric monoidal category $\mathbf{C}$. We construct this as an internal hom by considering for all $A \in \mathbf{C}$ adjoint pairs between $(-\otimes A) \dashv [A,-]: \mathbf{C}\to \mathbf{C}$, if they exist. If all these adjoint pairs exist the category is further called a closed monoidal category. 
	
	To see why this is the natural definition and not simply meaningless abstract nonsense it is useful to think about the construction over $\mathbf{Sets}$. 
	
	Consider the idea of set exponents (which internal homs are a special case of). Given sets $A, B$, we can consider the set $[A,B]$ of functions from $A \to B$ often denoted in terms of exponentiation $B^A= \prod_{a \in A} B$. This choice of notation stems from the fact that exponentiation gives the cardinality of the function set. Furthermore this leads to a well defined notion of exponential objects which we won't discuss here.
	
	But if we have any functions from a set $C$ into $B^A$ its naturally bijective to the functions from $C \times A \to B$ (note that cartesian products are also symmetric monoidal products so this is just a special case). Since we have the condition $(B^A)^C=B^{(A\times C)} $! 
	
	The adjoint definition is simply a rephrasing of this.
	
	\begin{proposition}[Correspondence of internal and external hom]
		In a symmetric monoidal category with internal homs we have a natural isomorphism \[ \Hom(A,[B,C]) \cong \Hom (A \times B, C)\]
	\end{proposition}
	
	\begin{proposition}[Tensor-Internal Hom adjunction]
		For a closed monoidal category $\mathbf C$ and objects $A,B,C \in \mathbf{C}$. The tensor product is left adjoint to the internal hom functor, in particular
		\[ [A\otimes B,C ] \cong [A, [B,C]] \]
	\end{proposition}
	
	We state a categorical fact here intstead of the appendix due to its huge importance, the proof is as in \cite{Awodey}
	\begin{proposition}[Right adjoints preserve limits]
	\end{proposition}
	\begin{proof}
		Say there is an adjunction pair between categories $\mathbf{C, D}$ given as $F \dashv U$, in particular $F: \mathbf{C} \rightleftarrows \mathbf{D}:U$, given some diagram $D:J \to \mathbf{D}$ where $J$ is some indexable category over which we are defining a diagram to find a limiting cone. Suppose the limit exists say $\lim_{\leftarrow} D_j$, note this is just an object in $\mathbf{D}$. Then for $A \in \mathbf{C}$ we have the following string of natural isormophisms,
		\begin{align*}
			\Hom_\mathbf{C}(A,U(\lim_{\leftarrow} D_j)) &\cong \Hom_D(F(A), \lim_{\leftarrow} D_j) \text{ ,by definition of adjunction}\\
			&\cong \lim_{\leftarrow} \Hom_\mathbf{D}(F(A), D_j) \text{ ,as Hom preserves limits}\\
			&\cong \lim_{\mathbf{C}}(A,U(D_j)) \text{ ,by adjunction}\\
			&\cong \Hom_{\mathbf{C}}(A, \lim_{\leftarrow} U (D_j))
		\end{align*}
		
		Now by Yoneda we can say $U(\lim_{\leftarrow} D_j) \cong \lim_{\leftarrow} U(D_j)$ which is the required result!
		
		Note that the dual construction holds too to state that right adjoints preserve colimits.
	\end{proof}
	This was proved in the appendix.
	
	Now we claim that this adjunction between Tensor product and internal homs is all we need to say that Tensor is left adjoint to the typical Hom functor in $\mathrm{Mod}$. 
	
	First note that $\mathrm{Mod}$ is a symmetric monoidal category with the typical tensor product. Now note that the typical Hom functor takes two modules $M, N$ to the set of module homomorphisms between them say $L(M,N)$ but this is a module itself! So the internal hom and external Hom functor correspond perfectly and one can be recovered from the other.
	
	We state these very quickly here
	Let $R$ be a commutative ring and $M,N$ modules over $R$.
	\begin{proposition}[Distribution over direct sums]
		\( M \otimes (\oplus_{i \in I}N_i) = \oplus_{i \in I} (M \otimes N_i)\)
	\end{proposition}
	\begin{proof}
		Direct sums in modules are by definition a coproduct of modules, whose universality condition gives that is is just a colimit of the following diagram.
		
		\[\begin{tikzcd}
			& \bullet \\
			\bullet && \bullet
			\arrow[from=2-1, to=1-2]
			\arrow[from=2-3, to=1-2]
		\end{tikzcd}\]
		
		Of course this means the direct sum distributes due to left adjoints preserving colimits.
	\end{proof}
	The notions of exact sequence will be seen again in the next section, but for completeness sake we state it here.
	\begin{definition}[Exact sequence of modules]
		A collection of modules $A_i$ and module homomorphisms $\alpha_i: A_i \to A_{i+1}$ is called exact if $\mathrm{Im}(\alpha_i) = \mathrm{Ker}(\alpha_{i+1})$
	\end{definition}
	\begin{definition}[Short exact sequence of modules] 
		If a sequence of modules and maps as such is exact, $0 \to A \to B \to C \to 0$. The map between $A \to B$ is injective and the map between $B \to C $ is surjective, furthermore $C \cong B/A$ due to the fact that $A $ can be considered an ideal of $B$ due to the exactness conditions.
	\end{definition}
		\begin{definition}[Split exact sequence of modules]
		If $B\cong A\oplus C$ in the above definition.
	\end{definition}
	All long exact sequences canonically decompose into short exact sequences. Assume $(A_\bullet, \alpha_\bullet)$ as in the first definition, it can be decomposed into the following collection of short exat sequences,
	\begin{align*}
		0 \to \mathrm{Im}\alpha_1 \to &A_1 \to \mathrm{Im} \alpha_2 \to 0\\
		0 \to \mathrm{Im} \alpha_2 \to &A_2 \to \mathrm{Im} \alpha_3 \to 0\\
		&\vdots
	\end{align*}
	
	\begin{definition}[Covariant exact functor]
		A covariant exact functor is a functor that preserves short exact sequences. If it preserves exactness of $0 \to A \to B \to C$ its called left exact and if it preserves exactness of $A \to B \to C \to 0$ it is called right exact. Simply switch left and right for contravariant case.
	\end{definition}
	
	\begin{proposition}[Right exactness of tensor product]
		
	\end{proposition}
	\begin{proof}
		This is simply a consequence of left adjoints preserving colimits.
	\end{proof}
	
	
	\begin{corollary}
		$R/I \otimes M = M/IM$ where $I$ an ideal of $R$
	\end{corollary}
	\begin{proof}
		Consider the short exact sequence $0 \to I \to R \to R/I \to 0$ as tensor product is right exact apply $-\otimes M$we get $I \otimes M \to R \otimes M \to (R/I) \otimes M \to 0$ but we know from above that $R \otimes M = M$ so we get $ I \otimes M \to M \to (R/I) \otimes M \to 0$.
		
		But by exactness we get $M/IM \cong (R/I) \otimes M/IM$
	\end{proof}
	
	\begin{corollary}
		$R/I \otimes R/J = R/(I+J)$ for $I,j$ ideals of $R$.
	\end{corollary}
	
	\section{Chain complexes}
	A \textbf{chain complex} $(A_\bullet, \varphi_\bullet)$ is a collection of modules over a commutative ring and homomorphisms $\varphi_i: A_i \to A_{i-1}$ such that $\varphi_i \varphi_{i+1}=0$,
	\[\begin{tikzcd}
		\cdots & {A_{i-1}} & {A_i} & {A_{i+1}} & \cdots
		\arrow["{\varphi_{i+1}}"', from=1-4, to=1-3]
		\arrow["{\varphi_i}"', from=1-3, to=1-2]
		\arrow["{\varphi_{i-1}}"', from=1-2, to=1-1]
		\arrow["{\varphi_{i+2}}"', from=1-5, to=1-4]
	\end{tikzcd}\]
	The \textbf{homology} of the complex at $A_i$ is denoted as its $i^{\mathrm{th}}$ homology defined as follows,
	\[ H_i := \ker \varphi_i/ \mathrm{im} \varphi_{i+1} \]
	Reversing the arrows gives us the analogous definitions for cochain complexes and cohomology.
	
	The homomorphisms are often called \textbf{`boundary operators'} or \textbf{`differentials'}. This nomenclature is motivated by the de Rahm cohomology. Furthermore elements of $\ker \varphi_i$ are called `\textbf{cycles}' and elements of $\mathrm{im} \varphi_{i+1}$ are called \textbf{boundaries}, this echoes the aphorism `cycles modulo boundaries' often encountered in singular homology.
	
	A chain complex is said to be \textbf{exact} if all its homologies are zero. In particular it is exact at one object if its homology there is zero.
	
	Note that these definitions can be easily viewed in terms of objects of Abelian categories, in particular for such a given enumerable chain it is called a $\Z-$graded chain complex. (WRITE THIS PROPERLY)
	
	\section{Free and Projective modules}
	Recall a\textbf{ free module} of rank $n$ is one that is isomorphic to $n$direct sums of its underlying ring. And homomorphisms from free modules to other modules are determined by the image of their generators, i.e. free objects are left adjoints to forgetful functors. \footnote{This holds in free monoids $\mathrm{Hom}_\mathbf{Mon}(F(X), M) \cong \mathrm{Hom}_\mathbf{Sets} (X, U(M))$ where $F(X)$ denotes the free monoid generated by elements from the set $X$ and $U(M)$ is the underlying set of a monoid $M$. Recall the hom-set definition of adjunctions. Refer to \cite[p. ~208]{Awodey} }
	
	A module $P$ is said to be \textbf{projective} if it satisfies the following lifting property, i.e. every morphism from $P$ to $N$ factors through an epi into $N$, note that the lift need not be unique this is \textit{not} an UMP
	\[\begin{tikzcd}
		&& M \\
		\\
		P && N
		\arrow[two heads, from=1-3, to=3-3]
		\arrow[from=3-1, to=3-3]
		\arrow[dashed, from=3-1, to=1-3]
	\end{tikzcd}\]
	\begin{lemma}[Free modules are projective]
	\end{lemma}
	\begin{proof}
		Consider the preimages of images of basis of $P$ in $N$, that lie in $M$. Then map basis elements from $P$ into these preimages.
	\end{proof}
	\begin{proposition}[Equivalent definitions of projectivity]
	TFAE,
	\begin{enumerate}
		\item $P$ is projective.
		\item For all epi's between $M\twoheadrightarrow N$, the induced map $\Hom(P,g):\mathrm{Hom}(P,M) \to \mathrm{Hom}(P,N)$ sending $f \mapsto g \circ f$ for $g:M \to N$ and $f:P \to M$ is an epi.
		\item For some epi from a free module $F$ to $P$, $\mathrm{Hom}(P,F) \to \mathrm{Hom}(P,P)$ is an epi.
		\item There exists $Q$ s.t. $P \oplus Q$ is free
		\item Short exact sequences of the form $0 \to A \to B \to P \to 0$ split, i.e. isomorphic to another short exact where middle term is $A \oplus P$ \footnote{In general any epis into projective objects split (i.e. have an inverse).}
	\end{enumerate}
	\end{proposition}
	\begin{proof}
		$1 \iff 2$ is restatement of definitions.
		
		$2 \implies 3$ also just substitution.
		
	$3 \implies 4$ consider a map in the preimage of identity in $\Hom(P,P)$ which is a splitting (inverse) of the epi $F$ into $P$,
	\[\begin{tikzcd}
		& P \\
		\\
		F && P
		\arrow["g", shift left=3, two heads, from=3-1, to=3-3]
		\arrow[""{name=0, anchor=center, inner sep=0}, "f"', from=1-2, to=3-1]
		\arrow[""{name=1, anchor=center, inner sep=0}, "{\mathrm{Id}_P=g \circ f}", dashed, two heads, from=1-2, to=3-3]
	\arrow[shorten <=6pt, shorten >=6pt, Rightarrow, from=0, to=1]
	\end{tikzcd}\]
	Now we have a short exact sequence $0 \to \ker g \to F \to P \to 0$, and also $f\circ g $ is idempotent so it naturally admits a decomposition $F = \image(f \circ g) \oplus \kernel (f \circ g)$\footnote{For some idempotent $e$, $1-e$ is also an idempotent and images under these two mappings decompose any module, furthermore image of $1-e$ is just kernel of $e$}=$\image (g) \oplus \kernel (g)$ the first by the 1st isomorphism theorem and the second by $f $ being a mono.
	
	$4 \implies 2$ simply as $\hom (P \oplus Q,-) = \hom(P,-) \oplus \hom(Q,-)$
	
	$1 \iff 5$ should be clear from above.
			
	\end{proof}
	
	\begin{theorem}[Proj. fin. generated modules over local rings are free]
	\end{theorem}
	\begin{proof}
			pick a minimal set of generators and see its residue classes in $M/\mathfrak{m}M$ as the basis of it as a vector space over $R/\mathfrak{m}$.
		
		Now as for some free module $F, F=\varphi(M)\oplus K$ for some $K$ and some homomorphism $\varphi: M \to F$, (by defn of projective module), 	we get \[ M/\mathfrak{m}M \cong 	F/\mathfrak{m}F = (R/\mathfrak{m})^n\cong R^n\otimes R/\mathfrak{m} \cong F \otimes R/\mathfrak{m} \cong (\varphi(M)\oplus K) \otimes R/\mathfrak{m}\]
		
		Finally we get $M/\mathfrak{m}M \cong M/\mathfrak{m}M \oplus K/\mathfrak{m}K\implies K=\mathfrak{m}K \implies K=0$ by Nakayama
	\end{proof}
	This holds for not necessarily finitely generated modules too refer to \cite[Th.~2.5]{matsumura_1987}	.
	\begin{theorem} If $M$ is a finitely presented module over a Noetherian ring $R$ (prime ideals fin gen) then TFAE
	\begin{enumerate}
		\item $M$ is projective.
		\item $M$ localized at maximal ideals is free.
		\item A finite set of elements $\{x_i\}^n$ in $R$ generate $R$ such that $M[x_i^{-1}]$ is free over $R[x_i^{-1}]$.
	\end{enumerate}		
	\end{theorem}
	\begin{proof}
	

%		Note If $R$ is local and $M$ is fin-gen projective module then $M$ is free, this is a consequence of Nakayama. As $M \oplus Q = R$ so if $R$ has maximal ideal $\mathfrak{m}$ then $M/\mathfrak{m}M$ is a vector space	over the field $R/\mathfrak{m}R$ and its basis lifts to minimal set of generators of $M$, consider $N=M/\sum R m_i$ and so $ N/ IN=M/(IM+\sum_i R m_i)=M/M=0\implies N=IN$ then apply typical Nakayama to get $N=0
%		\implies M= \sum_i R m_i$, for $I$ an ideal inside the Jacobson radical of $R$
%		
%		So now to prove $1 \iff 2 $ consider a finitely presented module localized over 
%		
%		If $M,N$ finitely presented over $R$ and their localizations are isomorphic then theres some element of $f\in R-P$ such that $M[f^{-1}] \cong N[f^{-1}]$

	\end{proof}

	\begin{theorem}[Quillen–Suslin]
		Every finitely generated projective module over a polynomial algebra is free.
	\end{theorem}
	

	This was an open problem for a long time as such the proof is very involved. Refer to \cite{nlab:quillen-suslin_theorem} or to a condensed proof in \cite[p. ~848]{lang02}

		\section{Resolutions}
	Given a module $M$ its \textbf{left resolution} is given by the data of a exact sequence $(A_\bullet, \varphi_\bullet)$ into $M$ as such,
	\[ 	\cdots \to A_1	\to A_0 \xrightarrow{\alpha} M \to 0 \]
	where $\alpha $ is called the \textbf{augmentation map}, if the exact sequence is free its a free resolution and such for projective. This will come up again in injective resolutions.
	
	\textbf{WRITE EXAMPLE OF KOSZKUL COMPLEX}
	
	
	\section{Injective modules}
	It is the dual notion to projective modules. For a commutative ring $R$ a $R-$module $Q$ is said to be injective if for every mono of $R-$ modules $M \rightarrowtail N$ and every homomorphism $ M \to Q$ lifts through the mono, as seen below,
		
	\[\begin{tikzcd}
		M && N \\
		\\
		Q
		\arrow[ tail, from=1-1, to=1-3]
		\arrow[ from=1-1, to=3-1]
		\arrow["{\exists}", dashed, from=1-3, to=3-1]
	\end{tikzcd}\]
	
	\begin{lemma}
		Product of injective modules is injective
	\end{lemma}
	
	\begin{proposition}[Baer's crieterion: Ideal inclusions are enough]
		If for every ideal $I \subseteq R$, homomorphisms $I \to Q$ lift through $R$ then $Q$ is injective, i.e.
		\[\begin{tikzcd}
			I && R \\
			\\
			Q
			\arrow["\subseteq", from=1-1, to=1-3]
			\arrow["", from=1-1, to=3-1]
			\arrow[dashed, from=1-3, to=3-1]
		\end{tikzcd}\]
	\end{proposition}
	\begin{proof}
		First consider two $R-$modules $M,N$ which have a mono between them $M \rightarrowtail N $ and say $f: M \to Q$ is some homomorphism. We have to find some lift $h: N \to Q$.
		
		Now consider a submodule $M'$ lying between $M,N$ and consider canonical extensions of $f$ (i.e. restriction on $N$ is just $f$). 
		\[\begin{tikzcd}
			M & {M'} & N \\
			\\
			Q
			\arrow[from=1-1, to=1-2]
			\arrow[from=1-2, to=1-3]
			\arrow["f"', from=1-1, to=3-1]
			\arrow["{f'}", from=1-2, to=3-1]
		\end{tikzcd}\]
		Arrange all possible submodules along with their morphisms into Q in a poset, sorted via inclusion. Now assuming choice, by Zorn's lemma we have a maximal element say $M'$ along with morphism $f': M' \to Q$ now if $M'=N$ we are done. 
		
		The proof proceeds via contradiction. Assume $M'\neq N$ now consider $x \in N-M'$ and consider the module $<x>+M'$. If we show there is a mapping from $\langle x \rangle+M' \to Q$ we contradict the choice of maximal element by Zorns and we are done.
		
		Consider the ideal $I=\{r \in R | rx \in M'\}$ of $R$ there is a morphism into $Q$ given by $g(r)=f'(rx)$ now by assumption this lifts over $R$ to some $k: R \to Q$. So for some arbitrary $z\in \langle x \rangle+M'$ which takes the form $rx+m$ we have the required extension of $\langle x \rangle+ M' \to Q$ as $f''(rx+m)=k(r)+f'(m)$ this contradicts maximality of the pair of $M'$ and $f'$ and so we are done.
		
		The assumption was wrong and $N=M'$.
		
	\end{proof}
	
	
	
	Baer's criterion has a few useful applications which we will now see. Note as Baer's relies on choice so do its corollaries.
	
	\begin{corollary}
		Abelian groups $Q$ regarded as $\Z-$modules are injective iff each element can be written as a product of nonzero $n \in \Z$ and some $q \in Q $ (i.e. $Q$ is divisible)
	\end{corollary}
	\begin{proof}
		$(\implies)$ If $Q$ is injective fix some arbitrary $q \in Q$. Then consider the map $f: \Z \to Q, f(1)=q$ and consider $g: \Z \rightarrowtail n\Z$ for some $n \neq 0$, now via injectivity theres a map from $h: n\Z \to Q$ such that $hg=f$ so $hg(1)=h(n)=nh(1)=f(1)=q$ so we got $n h(1)=q$ as required.
		
		$(\impliedby)$ Say $\forall q \in Q, q=nq', n\in \Z-\{0\}$. Consider $n\Z \subseteq \Z$ an ideal. And consider the canonical map $n\Z \to Q$ taking $n \to q$, but $q=n q'$ so the map $\Z \to Q$ taking $1 \to q'$ is the extension we require and by Baer's criterion we are done.
	\end{proof}
	The above result also holds to show that any subgroup of injective abelian group is abelian as if the abelian group is injective (i.e. divisible) so are its quotients, as such we have the particular case that $\Q/\Z$ is divisible. This forms an `almost' dual notion to Neilsen-Schierer.
	

	\begin{proposition}[$R-$Mod has enough injectives!]
		Every module has a mono into an injective module
	\end{proposition}
	\begin{proof}
	The proposition is proved in two parts. First we prove it for abelian groups (i.e. $\Z -$modules).
	
	For the first part note that $\Q $ and there is a canonical embedding from $\Z \rightarrowtail \Q$. Now every abelian group $A$ can be written as a quotient $F(A)/\ker f$
	where $f: A \to F(A)$ sends $A$ to its free group $F(A)$, which itself is a direct sum of copies of $Z$ so it embeds into direct sums of copies of $\Q $ which is divisble too.
	
	For the second part let $M \in R$-Mod now consider the homomorphisms $M \xrightarrow{\cong} \Hom_R(R,M) \to \Hom_\Z (R,M) \to \Hom_\Z(R, D)$ where $D$ is some divisble group as the above statement, and we consider $M$ as an abelian group forgetting structure in the last arrow. 
	
	We know $D$ is injective, now claim $\Hom_\Z(R,D)$ is an injective $R-$ module. Let $M \in R-$Mod. Now $\Hom_R(M,	\Hom_\Z(R,D) ) 	\cong \Hom_Z(M, D)$
	
	(Prove this categorically via free-forgetful adjunction)
	
	
%	Alternatively do this categorically, (Very confusing come back later)
%	
%	Now for the second part, in the general case we claim it holds for any category of $R-$modules, due to the free-forgetful adjunction. Namely that the forgetful functor $U: R\mathrm{-Mod}\to \mathrm{Ab}$ has the left adjoint of the `free' functor $F: \mathrm{Ab} \to R\mathrm{-Mod} $ characterized by sending $A \in \mathrm{Ab}$ to $R \otimes_\Z A$.
%	
%	So we have the following adjunction $F: \mathrm{Ab} \rightleftarrows  R\mathrm{-Mod}: U, F \dashv U$.	
%	
%	We claim if the free functor is exact and faithful then $R$-Mod has enough injectives. Consider $M \in R$-Mod since $\mathrm{Ab}$ has enough injectives there is some injective $A \in \mathrm{Ab}$ and a monomorphism $i:U(M) \rightarrowtail A$ the adjoint of this is a morphism $j:M \to F(A) $ so showing $F(A)$ is injective and $j$ being a monomorphism we are done.
	
	\end{proof}
%	\begin{corollary}
%		Direct sum of injective modules over Noetherian ring is injective
%	\end{corollary}
%	\begin{proof}
%		Say $R$ is a ring we know it is Noetherian if its ideals are finitely generated.
%		
%		To show that for a collection of injective modules $\{Q_i\}_{i \in I}$ over $R$ we have $Q=\bigoplus_{i \in I} Q_i $ is injective. It suffices to show due to Baer's criterion that for $f: I \subseteq R $ an ideal maps from $I \to Q $ lift through $R$.
%		
%		Now as $I$ is finitely generated say by elements $\{x_j\}_{j=1}^n$ consider $f_k=p_k \circ f$ where $p_k$ is the projection from $Q \to Q_k$ (note this is well defined as direct sum is a biproduct in modules).
%		\[\begin{tikzcd}
%			I && R \\
%			\\
%			{Q=\bigoplus_{i\in I} Q_i} & {Q_k}
%			\arrow["\subseteq", from=1-1, to=1-3]
%			\arrow["f"', from=1-1, to=3-1]
%			\arrow["{p_k}"', from=3-1, to=3-2]
%			\arrow["{f_k}", from=1-1, to=3-2]
%		\end{tikzcd}\]
%		So now summing across all $f_k$ we get $f$ factors through a finite summand of $Q$
%	\end{proof}

	\begin{proposition}[Bass-Papp]
		A ring is Noetherian iff every direct sum of injective modules over it are injective
	\end{proposition}
	\begin{proof}
		Let $R$ be a Noetherian ring. First note that due to right adjointness of the Hom functor we have $\Hom_R(M, \oplus_i N_i) \cong \oplus_i \Hom_R(M,N_i)$ for $M,N$ $R-$ modules.
	\end{proof}
	This result is also true for arbitrary direct limits as shown in \cite[Th. 3.46]{lam2012lectures}.
	
	
	\subsection{Injective resolutions}	
	\begin{definition}[Injective resolution]
		We can build an injective resolution inductively, starting from $M\in R-$Mod we may imbed it into an injective module say $Q_0$, $f: M \rightarrowtail Q_0$ now $\mathrm{coker}f=Q_0/f(M)$ into an injective module $Q_1$ and so on, giving us an \textbf{injective resolution}.
		\[ 0 \to M \to Q_0 \to Q_1 \to \dots \]
	\end{definition}
	The protoypical example is that of \[ 0 \to \Z \to \Q \to \Q/\Z \to 0 .\]
	The key point is that any module over arbitrary rings has a unique minimal injective resolution. This theory is developed with the notions of \textbf{injective envelopes}.
	
	\subsection{Injective envelopes}
	\begin{definition}[Essential submodules/extensions]
		For a ring $R$ and $M,E \in R-$Mod such that $M \subset N$ we say $M$ is an essential submodule of $E$ or $E$ is an essential extension of $M$ if every non-zero submodule of $E$ has nonzero intersection with $M$, i.e. $ M \cap L =0 \implies L =0$
\end{definition}

	The following proposition provides the definition of an \textbf{injective envelope}
	\begin{proposition}
		For a ring $R$
		\begin{enumerate}
			\item For $M, F \in R-$Mod and $M \subset F$ there is a maximal submodule $E $ of $F$ containing $M$ such that $M \subset E$ is essential
			\item If $F$ is injective so is $E$
			\item There is, up to isomorphism, a unique essential extension $E$ of $M$ that is an injective $R-$module, this $E$ is called the \textbf{injective envelope} of $M$ written as $E(M)$.
		\end{enumerate}
	\end{proposition}
	\begin{proof}
		We proof the propositions in order of statement,
		\begin{enumerate}
			\item Consider preorder of essential submodules $E_i$ of $F$. 
			Any submodule $N$ of $\cup_i E_i$ meets some $E_i$ nontrivially, and thus meets $M$ nontrivially. So $M$ is essential in $\cup E_i$. It follows via Zorn's that there is a maximal essential extension of $M $ contained in $F$.
			\item Suppose $F$ is injective and $M \subset E \subset F$ with $E$ beign the maximal essential extension of $M$. If $E'$ was an essential extension of $E$ in $F$ then it would be one of $M$ and so implies $E'=E$.
			
			It suffices to treat the case where $M=E$. Let $N$ be a submodule of $F$ maximal along those not meeting $E$. Since they dont meet $E \oplus N\cong E+N \subset F$. To show that $F=E+N$ from which it follows $E \oplus N \cong F$ so $E$ is injective.
			
			Consider $\alpha: E \subset F \twoheadrightarrow F/N$ an epi. As $N$ does not meet $E$, this epi is also a inclusion.
			
			(Now if a submodule of $F/N$ failed to meet $E$ then its preimage in $F$ would be submodule larger than $N$ not meeting $E$ which is contradiction.)
			
			
			Since $F$ is injective we may find a map $\beta: F/N \to F$ extending $\alpha.$ Since $\ker \beta \cap E = \ker \alpha =0$ and $E$ essential in $F/N$ we see even $\ker \beta =0$. So $\beta(F/N)$ essential extension of $E$ by maximality it means its equal to $E$ so $F/N=E$. So $E+N=F.$ And direct summand of injective is injective.
		\end{enumerate}
		
		
	\end{proof}
	\section{Flat modules}
	
	\section{Morphisms and homotopies of complexes}
	
	\subsection{Diagram chasing lemmas}
	
	\section{Derived functors}
	
	\subsection{Tor functor}
	\subsection{Ext functor}
	\subsection{Local cohomology}
	OVER HERE DO THE NEXT APPENDIX TOO
	
	\section{Mapping cones and double complex
	}
	\section{Spectral sequences}
	
	\section{Spectral sequence of double complex}
	
	\section{Derived categories}
	\section{Freyd-Mitchell Embedding}
	\begin{theorem}[Freyd-Mitchell Embedding]
		If $A$ is a small abelian category there is an exact full and faithful embedding functor from $A$ into the category of modules over some ring $R$ which embeds $A$ as a full subcategory.
	\end{theorem}
	\begin{proof}
		The inclusion from $A$ to its Ind-completion is full faithful, exact, compact, preserves generators and projectives.
		
		Locally finite presentable categories preserve injective cogenerators.
		
		Cocomplete abelian category with compact projective	generator is equivalent to End(P) modules
	\end{proof}
	
	\appendix
	\section{Categories}
	\subsection{Preliminary notions}
 
	A \textbf{category} consists of the following,
	\begin{itemize}
		\item Objects: A,B,C,\dots
		\item Arrows/Morphisms: f,g,h,\dots
		\item For each $ f $ there exists, $ \mathrm{dom}(f) , \mathrm{cod}(f)$ called domain and codomain of $ f $. We write $ f: A \to B $ to indicate $ A=\mathrm{dom}(f) $ and $ B=\mathrm{cod}(f) $.
		\item Given  $ f: A \to B$ and $ g: B \to C $ there exists, $ g \circ f: A \to C $ called the \textit{composite} of $ f $ and $ g $.
		\item For each $ A $, there exists $ 1_A:A\to A $ called the \textit{identity arrow} of $ A $.
		\item Arrows should also satisfy the following,
		\begin{itemize}
			\item Associativity: $ h \circ(g \circ f) = (h \circ g) \circ f,$ for all $ f:A \to B, g:B \to C, h: C \to D $.
			\item Unit: $ f\circ 1_A=f=1_B\circ f, $ for all $ f:A \to B $.	
		\end{itemize}
	\end{itemize}
	
	For categories $\mathbf{C}, \mathbf{D}$ we define a \textbf{functor} $ F: C \to D $ to be a a mapping of objects and arrows to objects and arrows, such that
	\begin{itemize}
		\item $ F(f:A\to B) =F(f):F(A)\to F(B)$
		\item $ F(1_a)=1_{F(A)} $
		\item $ F(g \circ f) = F(g)\circ F(f)$.
	\end{itemize}
	
		For an object $ A \in \mathbf{C} $ arbitrary arrows $ x:X\to A $ are called the \textbf{generalized elements} of $ A $ with stage of definition given by $ X $.
		
		A \textbf{bifunctor} is any functor of two variables i.e. domain in terms of a product category.
		\begin{lemma}[Bifunctor lemma]\label{bifunctorlemma}
		A mapping $F: \mathbf{A} \times \mathbf{B} \to \mathbf{C}$ is a bifunctor if its functorial in each component and the following square commutes,
		\[\begin{tikzcd}
			{A'} & {B'} && {F(A,B)} && {F(A,B')} \\
			\\
			A & B && {F(A',B)} && {F(A',B')}
			\arrow["{F(A,g)}", from=1-4, to=1-6]
			\arrow["{F(A',g)}"', from=3-4, to=3-6]
			\arrow["{F(f,B')}", from=1-6, to=3-6]
			\arrow["{F(f,B)}"', from=1-4, to=3-4]
			\arrow["g"{description}, from=3-2, to=1-2]
			\arrow["f"{description}, from=3-1, to=1-1]
		\end{tikzcd}\]
		\end{lemma}
		
		
		A \textbf{natural transformation} is a map between functors.
		
		For functors $F,G: \mathbf{C} \to \mathbf{D}$ a natural transformation $\eta: F \to G$ is a family of morphisms (in $\mathbf{D}$) which consist of \textbf{components} $\eta_X	$ which associates for every object $C \in \mathbf{C} $ a morphism between objects in $\mathbf{D}$, $\eta_C:F(C)\to G(C)$. Also components must commute naturally, in particular for $f: C \to C' $ we have $\eta_{C'} \circ F(f)=G(f)\circ\eta_C$, i.e. below diagram commutes,
		\[\begin{tikzcd}
			C && {F(C)} && {G(C)} \\
			\\
			{C'} && {F(C')} && {G(C')}
			\arrow["{\eta_C}", from=1-3, to=1-5]
			\arrow["{\eta_{C'}}", from=3-3, to=3-5]
			\arrow["{F(f)}"', from=1-3, to=3-3]
			\arrow["{G(f)}", from=1-5, to=3-5]
			\arrow["f"', from=1-1, to=3-1]
		\end{tikzcd}\]
		
		The \textbf{Yoneda embedding} is a functor $y:\mathbf{C}\to \mathbf{Sets}^{\mathbf{C}^\text{Op}}$ mapping objects to their contravariant representable functor (i.e. presheaves). In particular $y(C)=\textrm{Hom}_\mathbf{C}(-,C)$ it takes arrows to the natural transformation $yf=\mathrm{Hom}_\mathbf{C}(-,f):\mathrm{Hom}_\mathbf{C}(-,C) \to \mathrm{Hom}_\mathbf{C}(-,D)$
		
	\begin{proposition}[Yoneda Lemma]
			For a locally small category (i.e. small hom-sets) $\mathbf{C}$ we have $\textrm{Hom}(yC,F) \cong FC$ for $C \in \mathbf{C}$ and a functor $F \in \mathbf{Sets}^{\mathbf{C}^{op}}$. The isomorphism is natural in both $C$ and $F$.
	\end{proposition}
	
	For locally small categories
	\begin{itemize}
		\item The Yoneda embedding $y: \mathbf{C} \to \mathbf{Sets}^{\mathbf{C}^\mathbf{op}}$ is full and faithful.
		\item $yC \cong yC' \implies C \cong C'$ for objects $C,C'$
		\item All objects in presheaves are colimits of some representable functors (in particular the end). 
		\item So Yoneda embedding is the free cocompletion of any category. In particular there is a UMP for maps from $\mathbf{C}$ to any cocomplete categories factoring through presheaves wrt Yoneda embedding
	\end{itemize}	
	
	
	The functors $F: \mathbf{C} \rightleftarrows \mathbf{D}: U$ form an \textbf{adjunction} between categories if there exists a natural transformation $\eta: 1_\mathbf{C} \to U \circ F$ with this unit having the following UMP,
	\[\begin{tikzcd}
		D && {U(F(C))} && {U(D)} \\
		\\
		{F(C)} && C
		\arrow["f"', from=3-3, to=1-5]
		\arrow["{\eta_C}", from=3-3, to=1-3]
		\arrow["{U(g)}", from=1-3, to=1-5]
		\arrow[dashed, from=3-1, to=1-1]
	\end{tikzcd}\]
	
	$F$ is called the left adjoint of $U$ and vice versa, and is denoted as $F \dashv U$.
	
	Alternatively we also have the following formulation, $F \dashv U$ if for $C\in \mathbf{C}, D \in \mathbf{D}$ there exist natural isomorphisms $\phi: \textrm{Hom}_\mathbf{D}(FC, D) \cong \textrm{Hom}_\mathbf{C}(C, UD):\psi$
	
	Furthermore units and counits completely characterize adjoints, (WRITE WHY HERE).
		\subsection{Monoidal categories}
	A \textbf{monoidal category} is a category $\mathbf{C}$ with a bifunctor $ \otimes: \mathbf{C} \times \mathbf{C} \to \mathbf{C}$, a `unit' element $I$, and natural isomorphisms that make the functor associative and unital with $I$ as expected. It is a generalization of the notion of a `tensor product'. Its used to define enriched categories.
	
	This can be formalized as follows, there exists $I \in \mathbf{C}$ and natural isomorphisms,
	\begin{itemize}
		\item $\alpha_{ABC}: A \otimes (B \otimes C) \to (A \otimes B) \otimes C$
		\item $\lambda_A : I \otimes A \to A$ (read as left)
		\item $\rho_A: A \otimes I \to A$ (read as right)
	\end{itemize}
	
	And the following diagrams commute,
	\[\begin{tikzcd}
		& {(A\otimes B)\otimes (C \otimes D)} \\
		{A \otimes(B \otimes(C\otimes D))} && {((A\otimes B)\otimes C)\otimes D} \\
		{A \otimes ((B \otimes C) \otimes D)} && {(A \otimes (B\otimes C))\otimes D}
		\arrow[from=2-1, to=1-2]
		\arrow[from=1-2, to=2-3]
		\arrow[from=3-1, to=3-3]
		\arrow[from=3-3, to=2-3]
		\arrow[from=2-1, to=3-1]
	\end{tikzcd}\]
	\[\begin{tikzcd}
		{A\otimes(I\otimes B)} && {(A\otimes I)\otimes B} \\
		& {A\otimes B}
		\arrow[from=1-1, to=1-3]
		\arrow[from=1-3, to=2-2]
		\arrow[from=1-1, to=2-2]
	\end{tikzcd}\]
	
	These coherence conditions were first introduced by Maclane and a textbook presentation for it is there in his book \cite[Sec. ~VII Monoids]{lane1998categories}. The informal reasoning behind the coherence conditions is that all diagrams made with the given natural isomorphisms commute.
	
	We have a specific notion of a \textbf{symmetric monoidal category} wherein there exists another natural isomorphism called \textbf{braiding} which just `switches' the order of the tensor product, i.e. $B_{A,B}: A \otimes B \to B \otimes A  $ furthermore $B_{A,B}B_{B,A}=1_{A \otimes B}$, i.e. the tensor product is commutative!
	
	The coherence conditions for this includes two hexagon conditions which essentially say that braiding is associative across three objects as expected. This is excluded for brevity but can be seen in detail in \cite{nlab:braided_monoidal_category}	
	
	\subsection{External Hom and internal hom functor}
	The usual notion of the Hom functor involves simply considering the Hom-sets between two objects in a locally small category as a set. This is sometimes called the \textbf{external} Hom functor to differentiate it from the internal hom.
	\begin{definition}[Covariant Hom-functor]
		$\Hom(X,-): \mathbf{C} \to \mathbf{Sets}$ sending each object $Y \in \mathbf{C}$ to the set of morphisms between $X,Y$ and mapping morphisms naturally as composition, i.e., $\Hom(X,-)$ sending $f:Y \to Z$ to $\Hom(X,f): \Hom(X,Y)\to \Hom(X,Z)$
		sending $g\in \Hom(X,Y)$ to $g \circ f$. This has the natural covariant formulation.
	\end{definition}
	
	Perhaps more importantly, $\Hom(-,-)$ can be considered as a bifunctor in the natural way from $\mathbf{C}^{\mathrm{op}}\times \mathbf{C} \to \mathbf{Sets}$ \cite{nlab:hom-functor}.
	
	The notion of an \textbf{internal hom} is essentially a functor that behaves the same as a Hom functor but takes values into the category itself instead of into $\mathbf{Sets}$.
	Internal homs are well defined for categories which are symmetric monoidal category as follows.
	
	\begin{definition}[Internal hom]
		For a symmetric monoidal category ($\mathbf{C}, \otimes$), the internal hom is a functor \[ [X,-]: \mathbf{C}^{\mathrm{op}} \times \mathbf{C} \to \mathbf{C}\]
		characterized by a pair of adjoint functors $X \in \mathbf{C}$ \[ -\otimes X \dashv [X,-]: \mathbf{C}  \to \mathbf{C}\] If these exist it is called a symmetric closed monoidal category
	\end{definition}
	
	\begin{proposition}[Internal hom bifunctor]
		For a symmetric monoidal category ($\mathbf{C}, \otimes$), the internal hom is a unique bifunctor \[ [-,-]: \mathbf{C}^{\mathrm{op}} \times \mathbf{C} \to \mathbf{C}\]
		corresponds to the previous definition, and also \[  \Hom (A \otimes B, C) \cong \Hom[A, [B,C]]\] where this is the external hom
	\end{proposition}
	\begin{proof}
		The natural isomorphisms exist due to the homset definition of adjunction keeping the components fixed at $B$ and $B, C $. Furthermore bifunctorality is a consequence of bifunctor lemma \ref{bifunctorlemma}.
	\end{proof}
	
	
	\begin{theorem}[Tensor-Internal Hom adjunction]
		\[ [A\otimes B, C] \cong [A, [B,C]] \]
	\end{theorem}
	\begin{proof}
	\textbf{	requires full faithfull ness of yoneda complete this later}
	\end{proof}
	
	\begin{proposition}[Recovery of external hom from internal hom]
		\[ \Hom[A,B]\cong \Hom(I,[A,B]) \]
	\end{proposition}
	
	\subsection{Examples of internal homs}
	In general internal homs and external homs need not correspond, intuitively the internal hom contains a lot more information than just the external hom. And for monoidal unit $I$ maps from $I \to [A,B]$ correspond to the external homs $A \to B$.
	
	Heyting algebras correspond to intuitionistic propositional logic. And its internal hom corresponds to its preorder that is to say $p \leq q $ iff $  p \implies q$. This is quite interesting as external homs in Heyting algebras are simply sets with at most one element while internal homs are very illuminating. Furthermore extending this analogy to type theory the idea of the tensor internal hom adjunction is just the notion of `currying'. \textbf{(CITE WHY I FORGOT)}	.
	
	Example In the category of chain complexes and $I$ being the monoidal unit as before. Maps from $I \to A$ for $A$ a \textbf{(COMPLETE THIS)}
	
	\subsection{Enriched categories}
	Very often the hom-set of a category instead of just being a set may have additional structure (or we may want it to have additional structure). In order to formalize this notion we replace hom-sets with hom-\textit{objects} which themselves are objects in another category say $K$. We say its enriched over $K$. This choice of $K$ is typically taken to be monoidal in order to define composition of hom-sets via the monoidal product \footnote{This can further be dropped and work over bicategories \cite{garner2015enriched}. But this is out of the scope of this article.}. A category $\mathbf{C}$ is said to be enriched over some monoidal category $K$ if ordered pairs in $\mathbf{C}$ can be represented as objects in $K$ with composition defined via the monoidal product. The coherence conditions required to define this are excluded for brevity but can be seen in \cite[Chp. ~3]{riehl_2014} which presents it without proof.
	
	The actual definition isn't important here, we only mention it as the main object of study (abelian categories) is a special case of an enrichment over category of abelian groups.
	\subsection{Abelian Categories}
	We first define \textbf{pre-additive categories} as a category enriched over $\mathrm{Ab}$ with finite biproducts\footnote{Same as a direct sum, simultaneously a product and coproduct.} such that every morphism has a kernel and cokernel.
	
	But what are kernels and cokernels in the categorical sense? A \textbf{kernel} is a pullback of a morphism $f:A \to B$ and the unique morphism from $0 \to B$. Provided initials and pullbacks exist.
	
	\[\begin{tikzcd}
		{\ker f} && 0 \\
		\\
		A && B
		\arrow["f", from=3-1, to=3-3]
		\arrow[from=1-3, to=3-3]
		\arrow[from=1-1, to=1-3]
		\arrow[from=1-1, to=3-1]
	\end{tikzcd}\]
	The inuition behind this definition is that alternatively it is seen as an equalizer of a function $f$ and the unique morphism between $A \to 0 \to B$. The kernel object is the part of the domain that is 'going to zero'.
	
	Now we can define \textbf{abelian categories} as pre-additive categories wherein each mono is a kernel and each epic is a cokernel of some morphism and. Largely the purpose of abelian categories were motivated by wanting to generalize homological methods and to unify various (co)homology theories.
	
	\bibliographystyle{apalike}
	\bibliography{references}
	
	
\end{document}
