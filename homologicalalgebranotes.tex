\documentclass[12pt]{article}
\usepackage{graphicx}
\usepackage{amsmath}
\usepackage{fontawesome5}
\usepackage{booktabs}
\usepackage{amssymb}
\usepackage{amsthm}
\usepackage{lmodern}
\usepackage[english]{babel}
\usepackage[utf8x]{inputenc}
\usepackage[toc,page]{appendix}
\usepackage[nottoc]{tocbibind}
\numberwithin{equation}{section}
\graphicspath{ {./Images/} }
\usepackage[raggedright]{titlesec}
\usepackage{placeins}
\usepackage{tikz}
\usepackage{mathtools}
\usepackage{float}
\usepackage[autostyle]{csquotes}\usepackage{quiver}
\usepackage[activate={true,nocompatibility},final,tracking=true,kerning=true,spacing=true,factor=1100,stretch=10,shrink=10]{microtype}
\usepackage{hyperref}

\newcommand{\R}{\mathbb{R}}
\newcommand{\Q}{\mathbb{Q}}
\newcommand{\C}{\mathbb{C}}
\newcommand{\Z}{\mathbb{Z}}
\newcommand{\N}{\mathbb{N}}
\newcommand{\F}{\mathbb{F}}
\newcommand{\Hom}{{\mathrm{Hom}}}
\newcommand{\image}{{\mathrm{Im}}}
\newcommand{\kernel}{{\mathrm{Ker}}}
\newtheorem{theorem}{Theorem}[section]
\newtheorem{definition}{Definition}[section]
\newtheorem{corollary}{Corollary}[theorem]
\newtheorem{lemma}[theorem]{Lemma}

\newtheorem{proposition}{Proposition}[section]
%opening
\title{Sketches of homology}
\author{Bhoris Dhanjal}

\begin{document}
	\tableofcontents
	\maketitle
	These are compact notes on homological algebra. The material is presented mostly module theoretically with appropriate categorical abstractions when appropriate to better elucidate the concepts in play. The overall structure of the presented material is influenced by the appendix on homological algebra in Eisenbud's excellent book \cite{eisenbud2013commutative}.

	\section{Modules and tensor products}
	\begin{definition}[Module]
			Recall that for a commutative ring with unity $R$ a \textbf{module} over $R$ is a set $M$ along with operations of addition $+:M\times M \to M$ and scalar multiplication $R\times M \to M$ such that 
		\begin{itemize}
			\item $(M,+)$ is an abelian group,
			\item Scalar multiplication distributes over addition,
			\item Scalar multiplication respects unity and associativity.
		\end{itemize}
	\end{definition}
	
	\begin{definition}[Bilinear maps between modules]
		It is a map in two variable map which is linear over its arguments individually (when the other is held fixed).
	\end{definition}
	
	\begin{definition}[Tensor product]
			The \textbf{tensor product} for $R-$ modules $M,N$ denoted as $M \otimes_R N$ \footnote{When the choice of ring is obvious we will drop the subscript.} is an $R$ module along with a universal $A-$bilinear map $\varphi:	M \times N \to M \otimes N$. That is to say all $R-$bilinear maps factor through the tensor product, this is elucidated by saying the following diagram commutes for all $ R-$modules $P$ and $f$ some bilinear map from $M \times N$ into $P$
		\[\begin{tikzcd}
			{M\times N} && {M\otimes N} \\
			\\
			&& P
			\arrow["\varphi", from=1-1, to=1-3]
			\arrow["f"', from=1-1, to=3-3]
			\arrow["{\exists! \tilde{f}}", dashed, from=1-3, to=3-3]
		\end{tikzcd}\]
	\end{definition}
	

	
	In order to appreciate this universality first lets see the explicit construction of the tensor product.
	\subsection{Explicit construction of tensor products}
	The universality is the key point via which we aim to define a explicit construction. With it in mind the typical construction goes as such, consider the free module generated by the basis set $M \times N$. 
	
	Quotienting out the ideal which is generated by the bilinearity conditions we require will give us the explicit module we require. The required bilinearity conditions are,
	\begin{itemize}
		\item $(m+m', n)-(m,n)-(m',n')$ and $(m,n+n')-(m,n)-(m,n')$
		\item $(xm,n)-x(m,n)$ and $(m,xn)-x(m,n)$
	\end{itemize}
	For all $m,m'\in M, n,n' \in N, x \in R$. It is clear from the definitions that identifying these points together makes the quotient bilinear as required and it is the smallest such module as there we have added no extraneous conditions due to freeness. Furthermore any bilinear maps $f: M\times N \to P$ as above will extend naturally to $F: R^{\otimes M\times N} \to P$ and thus factoring through the tensor product mapping say $\varphi: M\times N \to M \otimes N$
	

	
	\subsection{Properties of tensor products}
	Let $R$ be some commutative ring with unity and let $M,N,P$ be $R-$modules.
	\begin{proposition}[Commutativity]
		$M \otimes N = N\otimes M$
	\end{proposition}
	\begin{proof}
		Note that the map $ (m,n)\to(m,n)$ induces a isomorphism between the free groups generated by the products. The conditions for bilinearity aren't affected so the quotients are the same.
	\end{proof}
	\begin{proposition}[Identity]
		$R \otimes M = M$
	\end{proposition}
	\begin{proof}
		Consider the canonical bilinear map $f: R \times M \to P$ given by $(r,m)\mapsto rm$. $M$ satisfies the UMP for this.
	\end{proof}
	\begin{proposition}[Associativity]
		$(M \otimes N) \otimes X = M \otimes (N \otimes X)$	
	\end{proposition}
	\begin{proof}
		Direct consequence of the UMP simply draw the diagrams first starting with $M \times N \to P$ and then $N \times X \to P$
	\end{proof}
	

	
	\subsection{Tensor-Internal Hom adjunction}
	As covered in the appendix we have a notion of an internal hom functor as a bifunctor $[-,-]:  \mathbf{C}^\mathrm{op} \times \mathbf{C} \to \mathbf{C}$ for some symmetric monoidal category $\mathbf{C}$. We construct this by considering for all $A \in \mathbf{C}$ adjoint pairs between $(-\otimes A) \dashv [A,-]: \mathbf{C}\to \mathbf{C}$, if they exist. If all these adjoint pairs exist the category is further called a closed monoidal category. 
	
	To see why this is the natural definition and not simply meaningless abstract nonsense it is useful to think about the construction over $\mathbf{Sets}$. 
	
	Consider the idea of set exponents (which internal homs are a special case of). Given sets $A, B$, we can consider the set $[A,B]$ of functions from $A \to B$ often denoted in terms of exponentiation $B^A= \prod_{a \in A} B$. This choice of notation stems from the fact that exponentiation gives the cardinality of the function set. Furthermore this leads to a well defined notion of exponential objects which we won't discuss here.
	
	But if we have any functions from a set $C$ into $B^A$ its naturally bijective to the functions from $C \times A \to B$ (note that cartesian products are also symmetric monoidal products so this is just a special case). This is very clearly seen in the condition $(B^A)^C=B^{(A\times C)} $! 
	
	The adjoint definition is simply a rephrasing of this.
	

	
	\begin{proposition}[Tensor-Internal Hom adjunction]
		For a closed monoidal category $\mathbf C$ and objects $A,B,C \in \mathbf{C}$. The tensor product is left adjoint to the internal hom functor, in particular
		\[ [A\otimes B,C ] \cong [A, [B,C]] \]
	\end{proposition}
	This was proved in the appendix in Th. \ref{tensorinternalhom} but in fact just Prop. \ref{hombifunctor} is enough for our case.
	
	We state a categorical fact here intstead of the appendix due to its huge importance, the proof is as in \cite{Awodey}
	\begin{proposition}[Right adjoints preserve limits]
	\end{proposition}
	\begin{proof}
		Say there is an adjunction pair between categories $\mathbf{C, D}$ given as $F \dashv U$, in particular $F: \mathbf{C} \rightleftarrows \mathbf{D}:U$, given some diagram $D:J \to \mathbf{D}$ where $J$ is some indexable category over which we are defining a diagram to find a limiting cone. Suppose the limit exists say $\lim_{\leftarrow} D_j$, note this is just an object in $\mathbf{D}$. Then for $A \in \mathbf{C}$ we have the following string of natural isormophisms,
		\begin{align*}
			\Hom_\mathbf{C}(A,U(\lim_{\leftarrow} D_j)) &\cong \Hom_D(F(A), \lim_{\leftarrow} D_j) \text{ ,by definition of adjunction}\\
			&\cong \lim_{\leftarrow} \Hom_\mathbf{D}(F(A), D_j) \text{ ,as Hom preserves limits}\\
			&\cong \lim_{\mathbf{C}}(A,U(D_j)) \text{ ,by adjunction}\\
			&\cong \Hom_{\mathbf{C}}(A, \lim_{\leftarrow} U (D_j))
		\end{align*}
		
		Now by Yoneda we can say $U(\lim_{\leftarrow} D_j) \cong \lim_{\leftarrow} U(D_j)$ which is the required result!
		
		Note that the dual construction holds too to state that right adjoints preserve colimits.
	\end{proof}
	
	
	Now we claim that this adjunction between Tensor product and internal homs is all we need to say that Tensor is left adjoint to the typical Hom functor in $\mathrm{Mod}$. 
	
	First note that $\mathrm{Mod}$ is a symmetric monoidal category with the typical tensor product. Now note that the typical Hom functor takes two modules $M, N$ to the set of module homomorphisms between them say $L(M,N)$ but this is a module itself! So the external Hom can be recovered from the internal hom. This is covered in the appendix in greater detail.
	
	We state these very quickly here
	Let $R$ be a commutative ring and $M,N$ modules over $R$.
	\begin{proposition}[Distribution over direct sums]
		\( M \otimes (\oplus_{i \in I}N_i) = \oplus_{i \in I} (M \otimes N_i)\)
	\end{proposition}
	\begin{proof}
		Direct sums in modules are by definition a coproduct of modules, whose universality condition gives that is is just a colimit of the following diagram.
		
		\[\begin{tikzcd}
			& \bullet \\
			\bullet && \bullet
			\arrow[from=2-1, to=1-2]
			\arrow[from=2-3, to=1-2]
		\end{tikzcd}\]
		
		Of course this means the direct sum distributes due to left adjoints preserving colimits.
	\end{proof}
	The notions of exact sequence will be seen again in the next section, but for completeness sake we state it here.
	\begin{definition}[Exact sequence of modules]
		A collection of modules $F_i$ and module homomorphisms $\alpha_i: F_i \to F_{i+1}$ is called exact if $\mathrm{Im}(\alpha_i) = \mathrm{Ker}(\alpha_{i+1})$
	\end{definition}
	\begin{definition}[Short exact sequence of modules] 
		If a sequence of modules and maps as such is exact, $0 \to A \to B \to C \to 0$. The map between $A \to B$ is injective and the map between $B \to C $ is surjective, furthermore $C \cong B/A$ due to the fact that $A $ can be considered an ideal of $B$ due to the exactness conditions.
		
	\end{definition}
		\begin{definition}[Split exact sequence of modules]
		If $B\cong A\oplus C$ in the above definition.
	\end{definition}
	All long exact sequences canonically decompose into short exact sequences. Assume $(A_\bullet, \alpha_\bullet)$ as in the first definition, it can be decomposed into the following collection of short exat sequences,
	\begin{align*}
		0 \to \mathrm{Im}\alpha_1 \to &A_1 \to \mathrm{Im} \alpha_2 \to 0\\
		0 \to \mathrm{Im} \alpha_2 \to &A_2 \to \mathrm{Im} \alpha_3 \to 0\\
		&\vdots
	\end{align*}
	
	\begin{definition}[Exact functor]
		A covariant exact functor is a functor that preserves short exact sequences. If it preserves exactness of $0 \to A \to B \to C$ its called left exact and if it preserves exactness of $A \to B \to C \to 0$ it is called right exact. 
	\end{definition}
	
	\begin{proposition}[Right exactness of tensor product]
		
	\end{proposition}
	\begin{proof}
		This is simply a consequence of left adjoints preserving colimits as it preserves cokernels and direct sums which are colimits so it is right exact.
		
		This is because if we have a short exact sequence \[ 0 \to A \xrightarrow{\alpha} B \xrightarrow{\beta} C \to 0\] and some functor which is right exact i.e.
		\[ F(A) \xrightarrow{F\alpha } F(B) \xrightarrow{F \beta } F(C) \to 0\]
		Firstly we know $\alpha $ is a mono and $\beta $ is an epi. So in particular $\mathrm{coker} \beta=0$ so in order for $ F \beta$ to be surjective we require $F(0)=0=F(\mathrm{coker} \beta )= \mathrm{coker}F(\beta)$
	 \end{proof}
	 Note that we can say tensor product is not a right adjoint as it is not left exact an example for this is the following,
	 \[ 0 \to \Z \xrightarrow{z \mapsto 2z} \Z \to \Z/2\Z \to 0 \]
	 Tensoring with $\otimes \Z/2\Z$ we expect \[ 0 \to \Z/2\Z \to \Z/2\Z \to \Z/2\Z  \] which is impossible to be exact as if it were we would have $\ker (\Z/2\Z \to \Z/2\Z ) = \image (\Z/2\Z \to \Z/2\Z )$.
	
	
	\begin{corollary}
		$R/I \otimes M = M/IM$ where $I$ an ideal of $R$
	\end{corollary}
	\begin{proof}
		Consider the short exact sequence $0 \to I \to R \to R/I \to 0$ as tensor product is right exact apply $-\otimes M$we get $I \otimes M \to R \otimes M \to (R/I) \otimes M \to 0$ but we know from above that $R \otimes M = M$ so we get $ I \otimes M \to M \to (R/I) \otimes M \to 0$.
		
		But by exactness we get $M/IM \cong (R/I) \otimes M/IM$
	\end{proof}
	
%	\begin{corollary}
%		$R/I \otimes R/J = R/(I+J)$ for $I,J$ ideals of $R$.
%	\end{corollary}
%	
	
	\section{Chain complexes}
	A \textbf{chain complex} $(F_\bullet, \varphi_\bullet)$ is a collection of modules over a commutative ring and homomorphisms $\varphi_i: F_i \to F_{i-1}$ such that $\varphi_i \varphi_{i+1}=0$,
\[\begin{tikzcd}
	\cdots & {F_{i+1}} & {F_i} & {F_{i-1}} & \cdots
	\arrow["{\varphi_{i+1}}", from=1-2, to=1-3]
	\arrow["{\varphi_i}", from=1-3, to=1-4]
	\arrow["{\varphi_{i-1}}", from=1-4, to=1-5]
	\arrow["{\varphi_{i+2}}", from=1-1, to=1-2]
\end{tikzcd}\]
	The \textbf{homology} of the complex at $F_i$ is denoted as its $i^{\mathrm{th}}$ homology defined as follows,
	\[ H_iF := \ker \varphi_i/ \mathrm{im} \varphi_{i+1} \]
	Reversing the arrows gives us the analogous definitions for cochain complexes and cohomology.
	
	The homomorphisms are often called \textbf{`boundary operators'} or \textbf{`differentials'}. This nomenclature is motivated by the de Rahm cohomology. Furthermore elements of $\ker \varphi_i$ are called `\textbf{cycles}' and elements of $\mathrm{im} \varphi_{i+1}$ are called \textbf{boundaries}, this echoes the aphorism `cycles modulo boundaries' often encountered in singular homology.
	
	A chain complex is said to be \textbf{exact} if all its homologies are zero. In particular it is exact at one object if its homology there is zero.
	
	Note that these definitions can be easily viewed in terms of objects of Abelian categories.
	
	\section{Free and Projective modules}
	Recall a\textbf{ free module} of rank $n$ is one that is isomorphic to $n$ direct sums of its underlying ring. And homomorphisms from free modules to other modules are determined by the image of their generators, i.e. free objects are left adjoints to forgetful functors. \footnote{This holds in free monoids $\mathrm{Hom}_\mathbf{Mon}(F(X), M) \cong \mathrm{Hom}_\mathbf{Sets} (X, U(M))$ where $F(X)$ denotes the free monoid generated by elements from the set $X$ and $U(M)$ is the underlying set of a monoid $M$, refer to \cite[p. ~208]{Awodey} }
	
	A module $P$ is said to be \textbf{projective} if it satisfies the following lifting property, every morphism from $P$ to $N$ factors through an epi into $N$. Note that the lift need not be unique this is \textit{not} an UMP
	\[\begin{tikzcd}
		&& M \\
		\\
		P && N
		\arrow[two heads, from=1-3, to=3-3]
		\arrow[from=3-1, to=3-3]
		\arrow[dashed, from=3-1, to=1-3]
	\end{tikzcd}\]
	\begin{lemma}[Free modules are projective]
	\end{lemma}
	\begin{proof}
		Consider the preimages of images of basis of $P$ in $N$, that lie in $M$. Then map basis elements from $P$ into these preimages.
	\end{proof}
	\begin{proposition}[Equivalent definitions of projectivity]
	TFAE,
	\begin{enumerate}
		\item $P$ is projective.
		\item For all epi's between $M\twoheadrightarrow N$, the induced map $\Hom(P,g):\mathrm{Hom}(P,M) \to \mathrm{Hom}(P,N)$ sending $f \mapsto g \circ f$ for $g:M \to N$ and $f:P \to M$ is an epi.
		\item For some epi from a free module $F$ to $P$, $\mathrm{Hom}(P,F) \to \mathrm{Hom}(P,P)$ is an epi.
		\item There exists $Q$ s.t. $P \oplus Q$ is free
		\item Short exact sequences of the form $0 \to A \to B \to P \to 0$ split, i.e. isomorphic to another short exact where middle term is $A \oplus P$ \footnote{In general any epis into projective objects split (i.e. have an inverse).}
	\end{enumerate}
	\end{proposition}
	\begin{proof}
		$1 \iff 2$ is restatement of definitions.
		
		$2 \implies 3$ also just substitution.
		
	$3 \implies 4$ consider a map in the preimage of identity in $\Hom(P,P)$ which is a splitting (inverse) of the epi $F$ into $P$,
	\[\begin{tikzcd}
		& P \\
		\\
		F && P
		\arrow["g", shift left=3, two heads, from=3-1, to=3-3]
		\arrow[""{name=0, anchor=center, inner sep=0}, "f"', from=1-2, to=3-1]
		\arrow[""{name=1, anchor=center, inner sep=0}, "{\mathrm{Id}_P=g \circ f}", dashed, two heads, from=1-2, to=3-3]
	\arrow[shorten <=6pt, shorten >=6pt, Rightarrow, from=0, to=1]
	\end{tikzcd}\]
	Now we have a short exact sequence $0 \to \ker g \to F \to P \to 0$, and also $f\circ g $ is idempotent so it naturally admits a decomposition $F = \image(f \circ g) \oplus \kernel (f \circ g)$\footnote{For some idempotent $e$, $1-e$ is also an idempotent and images under these two mappings decompose any module, furthermore image of $1-e$ is just kernel of $e$}=$\image (g) \oplus \kernel (g)$ the first by the 1st isomorphism theorem and the second by $f $ being a mono.
	
	$4 \implies 2$ simply as $\hom (P \oplus Q,-) = \hom(P,-) \oplus \hom(Q,-)$
	
	$1 \iff 5$ should be clear from above.
			
	\end{proof}
	
	\begin{theorem}[Proj. fin. generated modules over local rings are free]
	\end{theorem}
	\begin{proof}
			pick a minimal set of generators and see its residue classes in $M/\mathfrak{m}M$ as the basis of it as a vector space over $R/\mathfrak{m}$.
		
		Now as for some free module $F, F=\varphi(M)\oplus K$ for some $K$ and some homomorphism $\varphi: M \to F$, (by defn of projective module), 	we get \[ M/\mathfrak{m}M \cong 	F/\mathfrak{m}F = (R/\mathfrak{m})^n\cong R^n\otimes R/\mathfrak{m} \cong F \otimes R/\mathfrak{m} \cong (\varphi(M)\oplus K) \otimes R/\mathfrak{m}\]
		
		Finally we get $M/\mathfrak{m}M \cong M/\mathfrak{m}M \oplus K/\mathfrak{m}K\implies K=\mathfrak{m}K \implies K=0$ by Nakayama
	\end{proof}
	This holds for not necessarily finitely generated modules too refer to \cite[Th.~2.5]{matsumura_1987}	.
	\begin{proposition} If $M$ is a finitely presented module over a Noetherian ring $R$ (prime ideals fin gen) then TFAE
	\begin{enumerate}
		\item $M$ is projective.
		\item $M$ localized at maximal ideals is free.
		\item A finite set of elements $\{x_i\}^n$ in $R$ generate $R$ such that $M[x_i^{-1}]$ is free over $R[x_i^{-1}]$.
	\end{enumerate}		
	\end{proposition}
This proceeds just from the previous result.

%		Note If $R$ is local and $M$ is fin-gen projective module then $M$ is free, this is a consequence of Nakayama. As $M \oplus Q = R$ so if $R$ has maximal ideal $\mathfrak{m}$ then $M/\mathfrak{m}M$ is a vector space	over the field $R/\mathfrak{m}R$ and its basis lifts to minimal set of generators of $M$, consider $N=M/\sum R m_i$ and so $ N/ IN=M/(IM+\sum_i R m_i)=M/M=0\implies N=IN$ then apply typical Nakayama to get $N=0
%		\implies M= \sum_i R m_i$, for $I$ an ideal inside the Jacobson radical of $R$
%		
%		So now to prove $1 \iff 2 $ consider a finitely presented module localized over 
%		
%		If $M,N$ finitely presented over $R$ and their localizations are isomorphic then theres some element of $f\in R-P$ such that $M[f^{-1}] \cong N[f^{-1}]$
	\begin{theorem}[Quillen–Suslin]
		Every finitely generated projective module over a polynomial algebra is free.
	\end{theorem}
	

	This was an open problem for a long time as such the proof is very involved. Refer to \cite{nlab:quillen-suslin_theorem} or to a condensed proof in \cite[p. ~848]{lang02}

		\section{Resolutions}
	Given a module $M$ its \textbf{left resolution} is given by the data of a exact sequence $(A_\bullet, \varphi_\bullet)$ into $M$ as such,
	\[ 	\cdots \to A_1	\to A_0 \xrightarrow{\epsilon} M \to 0 \]
	where $\epsilon $ is called the \textbf{augmentation map}, if the exact sequence is free its a free resolution and such for projective. 
	
	If we have a cochain complex instead it forms a \textbf{right resolution} and if its elements are injective we call them injective resolutions.
	
	\textbf{TO DO: KOSZKUL COMPLEX AND HILBERT SYZYGY}
	We previously saw the definition of a bilinear map when discussing tensor products. Now consider $n$- linear maps for some map between $R$-Modules $M,P$. Repeated application of the tensor product still provides us with universal such module, $M^n \to \otimes_{i=1}^n M$.
	
	Also a map is called $n$-alternating if it vanishes when two of the arguments are the same. This also implies that sign changes when the arguments are interchanged. Also for a permutation the sign change changes to the sign of the permutation. Now when we require a notion of a universal $n$-linear alternating map we arrive to the definition of a wedge product.
	
	In particular for a specific $n$. There is a universal alternating $n$-linear map sending $M \to \Lambda^n M$. If $M$ is finitely generated by $r$ elements then $\Lambda M^n$ is a free module of rank $\binom{r}{n}$.
	
	\begin{definition}[Tensor algebra]
		For a $R$-Module $M$ and $T^n M = \otimes^n M$ the tensor algebra is defined to be \[ T(M)=\oplus_{n=0}^\infty T^n M \]
	\end{definition}
	
	
	
	\begin{definition}[Exterior algebra and wedge product]
		For an $R$-Module $M$. Consider the Tensor algebra $T(M)=\oplus_n \otimes^n M	$. Consider the ideal $I$ spanned by elements $v \otimes v$ for $v \in T(M)$.
		
		The quotient of $T(V)/I := \Lambda(M) $ is called the exterior algebra and its product the wedge product.	It is universal with respect to multilinear alternating maps.
		
		For $v \Lambda^nM, w \in \Lambda^m M$ multiplication is defined as such,
		\[ v \Lambda w = (-1)^{nm} w \Lambda v\]
	\end{definition}
	
	
	\begin{definition}[Koszkul complex]
		For a $R$-module $M$ there is an associated chain complex of $n^\textbf{th}$ exterior algebras
		\[ \Lambda^n M \to \Lambda^{n-1} M \to \cdots \to \Lambda^0 M \cong M \]
	\end{definition}
	
	
	\begin{definition}[Regular sequence]
		A sequence of elements $r_1,\cdots,r_n$ is said to be a regular sequence in $R$ if $r_1$ is not a zero divisor of $R$, $r_2$ is not a zero divisor of $R/\rangle r_1\rangle $, and so on
	\end{definition}
	\begin{proposition}
		For a finite regular sequence $\{r_i\}_{i=1}^n$ of a ring $R$ the Koszkul complex forms the canonical free resolution of $R/\langle r_1,\cdots,r_n\rangle$ of the form
			\[ 0 \to R^{\binom{n}{n}} \to \cdots \to R^{\binom{n}{1}} \to R \to R/\langle r_1,\cdots, r_n \rangle \to 0\]
	\end{proposition}
	\begin{proof}
		TO DO
	\end{proof}
	
	\section{Injective modules}
	It is the dual notion to projective modules. For a commutative ring $R$ a $R-$module $Q$ is said to be injective if for every mono of $R-$ modules $M \rightarrowtail N$ and every homomorphism $ M \to Q$ lifts through the mono, as seen below,
		
	\[\begin{tikzcd}
		M && N \\
		\\
		Q
		\arrow[ tail, from=1-1, to=1-3]
		\arrow[ from=1-1, to=3-1]
		\arrow["{\exists}", dashed, from=1-3, to=3-1]
	\end{tikzcd}\]
	
	\begin{lemma}
		Product of injective modules is injective
	\end{lemma}
	
	\begin{proposition}[Baer's crieterion: Ideal inclusions are enough]
		If for every ideal $I \subseteq R$, homomorphisms $I \to Q$ lift through $R$ then $Q$ is injective, i.e.
		\[\begin{tikzcd}
			I && R \\
			\\
			Q
			\arrow["\subseteq", from=1-1, to=1-3]
			\arrow["", from=1-1, to=3-1]
			\arrow[dashed, from=1-3, to=3-1]
		\end{tikzcd}\]
	\end{proposition}
	\begin{proof}
		First consider two $R-$modules $M,N$ which have a mono between them $M \rightarrowtail N $ and say $f: M \to Q$ is some homomorphism. We have to find some lift $h: N \to Q$.
		
		Now consider a submodule $M'$ lying between $M,N$ and consider canonical extensions of $f$ (i.e. restriction on $N$ is just $f$). 
		\[\begin{tikzcd}
			M & {M'} & N \\
			\\
			Q
			\arrow[from=1-1, to=1-2]
			\arrow[from=1-2, to=1-3]
			\arrow["f"', from=1-1, to=3-1]
			\arrow["{f'}", from=1-2, to=3-1]
		\end{tikzcd}\]
		Arrange all possible submodules along with their morphisms into Q in a poset, sorted via inclusion. Now assuming choice, by Zorn's lemma we have a maximal element say $M'$ along with morphism $f': M' \to Q$ now if $M'=N$ we are done. 
		
		The proof proceeds via contradiction. Assume $M'\neq N$ now consider $x \in N-M'$ and consider the module $<x>+M'$. If we show there is a mapping from $\langle x \rangle+M' \to Q$ we contradict the choice of maximal element by Zorns and we are done.
		
		Consider the ideal $I=\{r \in R | rx \in M'\}$ of $R$ there is a morphism into $Q$ given by $g(r)=f'(rx)$ now by assumption this lifts over $R$ to some $k: R \to Q$. So for some arbitrary $z\in \langle x \rangle+M'$ which takes the form $rx+m$ we have the required extension of $\langle x \rangle+ M' \to Q$ as $f''(rx+m)=k(r)+f'(m)$ this contradicts maximality of the pair of $M'$ and $f'$ and so we are done.
		
		The assumption was wrong and $N=M'$.
		
	\end{proof}
	
	
	
	Baer's criterion has a few useful applications which we will now see. Note as Baer's relies on choice so do its corollaries.
	
	\begin{corollary}
		Abelian groups $Q$ regarded as $\Z-$modules are injective iff each element can be written as a product of nonzero $n \in \Z$ and some $q \in Q $ (i.e. $Q$ is divisible)
	\end{corollary}
	\begin{proof}
		$(\implies)$ If $Q$ is injective fix some arbitrary $q \in Q$. Then consider the map $f: \Z \to Q, f(1)=q$ and consider $g: \Z \rightarrowtail n\Z$ for some $n \neq 0$, now via injectivity theres a map from $h: n\Z \to Q$ such that $hg=f$ so $hg(1)=h(n)=nh(1)=f(1)=q$ so we got $n h(1)=q$ as required.
		
		$(\impliedby)$ Say $\forall q \in Q, q=nq', n\in \Z-\{0\}$. Consider $n\Z \subseteq \Z$ an ideal. And consider the canonical map $n\Z \to Q$ taking $n \to q$, but $q=n q'$ so the map $\Z \to Q$ taking $1 \to q'$ is the extension we require and by Baer's criterion we are done.
	\end{proof}
	The above result also holds to show that any subgroup of injective abelian group is abelian as if the abelian group is injective (i.e. divisible) so are its quotients, as such we have the particular case that $\Q/\Z$ is divisible. This forms an `almost' dual notion to Neilsen-Schierer.
	

	\begin{proposition}[$R-$Mod has enough injectives!]
		Every module has a mono into an injective module
	\end{proposition}
	\begin{proof}
	The proposition is proved in two parts. First we prove it for abelian groups (i.e. $\Z -$modules).
	
	For the first part note that $\Q $ and there is a canonical embedding from $\Z \rightarrowtail \Q$. Now every abelian group $A$ can be written as a quotient $F(A)/\ker f$
	where $f: A \to F(A)$ sends $A$ to its free group $F(A)$, which itself is a direct sum of copies of $Z$ so it embeds into direct sums of copies of $\Q $ which is divisble too.
	
	For the second part let $M \in R$-Mod now consider the homomorphisms $M \xrightarrow{\cong} \Hom_R(R,M) \to \Hom_\Z (R,M) \to \Hom_\Z(R, D)$ where $D$ is some divisble group as the above statement, and we consider $M$ as an abelian group forgetting structure in the last arrow. 
	
	We know $D$ is injective, now claim $\Hom_\Z(R,D)$ is an injective $R-$ module. Let $M \in R-$Mod. Now $\Hom_R(M,	\Hom_\Z(R,D) ) 	\cong \Hom_Z(M, D)$
	
	(Prove this categorically via free-forgetful adjunction)
	
	
%	Alternatively do this categorically, (Very confusing come back later)
%	
%	Now for the second part, in the general case we claim it holds for any category of $R-$modules, due to the free-forgetful adjunction. Namely that the forgetful functor $U: R\mathrm{-Mod}\to \mathrm{Ab}$ has the left adjoint of the `free' functor $F: \mathrm{Ab} \to R\mathrm{-Mod} $ characterized by sending $A \in \mathrm{Ab}$ to $R \otimes_\Z A$.
%	
%	So we have the following adjunction $F: \mathrm{Ab} \rightleftarrows  R\mathrm{-Mod}: U, F \dashv U$.	
%	
%	We claim if the free functor is exact and faithful then $R$-Mod has enough injectives. Consider $M \in R$-Mod since $\mathrm{Ab}$ has enough injectives there is some injective $A \in \mathrm{Ab}$ and a monomorphism $i:U(M) \rightarrowtail A$ the adjoint of this is a morphism $j:M \to F(A) $ so showing $F(A)$ is injective and $j$ being a monomorphism we are done.
	
	\end{proof}
Alternatively consider the following alternative proof for the second part
\begin{proposition}[Categorical proof]
	$R-$Mod has enough injectives.
\end{proposition}
\begin{proof}
	We assume that $\mathrm{Ab}$ has enough injectives as proven via the divisibility condition.
	
	The proof proceeds categorically and relies on the notion of the free-forgetful functor adjunction. In particular in this case the forgetful functor $U: R\mathrm{-Mod} \to \mathrm{Ab}$ formed by restriction of scalars i.e. sending the module to its underlying abelian group. To put this analytically consider $f: \Z \to R$. Now a $R$-Module $M$ can be canonically taught of as an $\Z$ module due to the following morphism $M \times \Z \to M$ given by $(m,z) \mapsto mf(z)$.
	
	 Now suppose $M,N \in R$-Mod and some $u:M \to N$ is an $R-$homomorphism this naturally maps to a $\Z-$homomorphism as $u(mz)=u(m)z$. This describes the notion of `forgetting' the module structure.
	
	This forgetful functor has both left and right adjoint. With its left adjoint being sending $A \in \mathrm{Ab} $ to $R \otimes_\Z A$ (this is the free functor) and its right adjoint being sending $A\in \mathrm{Ab}$ to $\Hom_\Z(R,A)$.
	
	This is a specific case of the extension $\dashv $ restriction $\dashv$ coextension adjoint triple.
	
	The point being that as the forgetful functor has both left and right adjoints, it preserves both limits and colimits, in particular kernels and cokernels. So it is both left and right exact therefore it is an exact functor. It is also clearly a faithful functor. Now due to Prop. \ref{adjointinjective} we have that $R-$mod has enough injectives.
\end{proof}

	\begin{proposition}[Bass-Papp]
		A ring is Noetherian iff every direct sum of injective modules over it are injective
	\end{proposition}
	\begin{proof}
		Let $R$ be a Noetherian ring. First note that for finitely generated $M$ we have $\Hom_R(M, \oplus_i N_i) \cong \oplus_i \Hom_R(M,N_i)$ for $M,N$ $R-$ modules.
		
		($\implies $)Let $R$ be Noetherian and let $E_i $ be a family of injective $R$-Modules. Note that $I \subset R$ ideals of $R$ are finitely generated. Now we also have the surjection $ \Hom(R,E_i) \twoheadleftarrow \Hom(I,E_i) $ by inclusion. The ideals of a Noetherian ring are finitely generated so by the above condition we also get a surjection between $\oplus_i \Hom (R, E_i) \cong \Hom(R, \oplus_i E_i) \twoheadleftarrow \Hom(I, \oplus_i E_i)$ now Baer's criterion implies that $\oplus_i E_i$ is injective.
		
		($\impliedby $) Suppose if possible $R$ is not Noetherian so we have a non stable ascending chain of ideals $\{J_i\}_{i \in \Lambda}$. Consider its union say $J= 	\cup_{i \in \Lambda} J_i$. And consider homomorphism $J \to \oplus_i E(R/J_i)$ where $E(-)$ denotes injective envelope (discussed later). Claim that the homomorphism doesn't extend over $R$ (Why complete this?)
	\end{proof}
	This result is also true for arbitrary direct limits as shown in \cite[Th. 3.46]{lam2012lectures}.
	\begin{definition}[Indecomposable module]
			An injective module $E$ is indecomposable if it cannot be written as a direct sum of two nonzero modules.
	\end{definition}
	
	\begin{proposition}[Matlis' Theorem]
		If $R$ is Noetherian, and $E$ is some indecomposable injective $R-$module $M$ we have,
		\begin{enumerate}
			\item $M \cong E(R/P)$ for some prime ideal $P\subset R$.
			\item $E(R/P)\cong E(R/Q) \iff P=Q$ so there is a 1-1 correspondence between indecomposable injectives and primes.
		\end{enumerate} 
	\end{proposition}
	\begin{proof}
		Here $E(R/P)$ denotes the injective envelope of $R/P$.	
	\end{proof}
	
	\subsection{Injective resolutions}	
	\begin{definition}[Injective resolution]
		We can build an injective resolution inductively, starting from $M\in R-$Mod we may imbed it into an injective module say $Q_0$, $f: M \rightarrowtail Q_0$ now $\mathrm{coker}f=Q_0/f(M)$ into an injective module $Q_1$ and so on, giving us an \textbf{injective resolution}.
		\[ 0 \to M \to Q_0 \to Q_1 \to \dots \]
	\end{definition}
	The protoypical example is that of \[ 0 \to \Z \to \Q \to \Q/\Z \to 0 .\]
	The key point is that any module over arbitrary rings has a unique minimal injective resolution. This theory is developed with the notions of \textbf{injective envelopes}.
	
	\subsection{Injective envelopes}
	\begin{definition}[Essential submodules/extensions]
		For a ring $R$ and $M,E \in R-$Mod such that $M \subset N$ we say $M$ is an \textbf{essential submodule} of $E$ or $E$ is an \textbf{essential extension} of $M$ if every non-zero submodule of $E$ has nonzero intersection with $M$, i.e. $ M \cap L =0 \implies L =0$
\end{definition}

	The following proposition provides the definition of an \textbf{injective envelope}
	\begin{proposition}
		For a ring $R$
		\begin{enumerate}
			\item For $M, F \in R-$Mod and $M \subset F$ there is a maximal submodule $E $ of $F$ containing $M$ such that $M \subset E$ is essential
			\item If $F$ is injective so is $E$
			\item There is, up to isomorphism, a unique essential extension $E$ of $M$ that is an injective $R-$module, this $E$ is called the \textbf{injective envelope} of $M$ written as $E(M)$.
		\end{enumerate}
	\end{proposition}
	\begin{proof}
		We proof the propositions in order of statement,
		\begin{enumerate}
			\item Consider preorder of essential submodules $E_i$ of $F$. 
			Any submodule $N$ of $\cup_i E_i$ meets some $E_i$ nontrivially, and thus meets $M$ nontrivially. So $M$ is essential in $\cup E_i$. It follows via Zorn's that there is a maximal essential extension of $M $ contained in $F$.
			\item Suppose $F$ is injective and $M \subset E \subset F$ with $E$ being the maximal essential extension of $M$. If $E'$ was an essential extension of $E$ in $F$ then it would be one of $M$ and so implies $E'=E$.
			
			 Let $N$ be a submodule of $F$ maximal along those not meeting $E$. Since they dont meet $E \oplus N\cong E+N \subset F$. To show that $F=E+N$ from which it follows $E \oplus N \cong F$ so $E$ is injective.
			
			Consider $\alpha: E \subset F \twoheadrightarrow F/N$ an epi. As $N$ does not meet $E$, this epi is also a inclusion.
			
			(Now if a submodule of $F/N$ failed to meet $E$ then its preimage in $F$ would be submodule larger than $N$ not meeting $E$ which is contradiction.)
			
			
			Since $F$ is injective we may find a map $\beta: F/N \to F$ extending $\alpha.$ Since $\ker \beta \cap E = \ker \alpha =0$ and $E$ essential in $F/N$ we see even $\ker \beta =0$. So $\beta(F/N)$ essential extension of $E$ by maximality it means its equal to $E$ so $F/N=E$. So $E+N=F.$ And direct summand of injective is injective.
			\item By the notion of enough injectives we can say there is a mono from $M$ to $E$ by previous construction. To show uniqueness suppose there are two such essential maximal inclusions. So there are monos and an extension (due to essentiality of $E_2$)as below,
				\[\begin{tikzcd}
					& {E_1} \\
					M & {E_2}
					\arrow["{f_1}", tail, from=2-1, to=1-2]
					\arrow["{f_2}"', tail, from=2-1, to=2-2]
					\arrow["g", from=1-2, to=2-2]
				\end{tikzcd}\]
				We get $\ker g=0$ as $\ker f_1=0$ and chasing $M$ around kernel of $g$ restricted to $M$ is also 0. So we get $g(E_1) $ injective submodule of $E_2$ and its a direct summand say $E_2=g(E_1) \oplus K$ but also $f_2(M)\subset g(E_1)$ so by essentiality of $M$ in $E_2$ again $K=0$ so they are the same.
			
		\end{enumerate}
		
	
	\end{proof}
	To recap,
	\begin{definition}[Injective envelope/hull]
		For $R-$modules $M \subset F$ the unique essential extension of $M$	which is injective and as an intermediate submodule is called the injective envelope of $M$.
	\end{definition}
	The injective hull can be used to define a unique minimum injective resolution. (\textbf{???})
	\begin{proposition}
		If $N \subset M$ is an essential submodule then any map $M \to X$ that restricts to a monomorphism on $N$ is a monomorphism.
	\end{proposition}
	\begin{proof}
		Consider the injective envelope of $N$ say $E$.
		\[\begin{tikzcd}
			N \\
			E \\
			M && X
			\arrow["\subseteq"', from=1-1, to=2-1]
			\arrow["\subseteq"', from=2-1, to=3-1]
			\arrow[from=3-1, to=3-3]
			\arrow[tail, from=1-1, to=3-3]
			\arrow[dashed, from=3-3, to=2-1]
		\end{tikzcd}\]
	\end{proof}
%	\section{Flat modules}
	\section{More on chain complexes}
	\subsection{Morphisms and homotopies of complexes}
	Recall the definition of a chain complex.
	A \textbf{chain complex} $(F_\bullet, \varphi_\bullet)$ is a collection of $R-$Modules and homomorphisms $\varphi_i: F_i \to F_{i-1}$ such that $\varphi_i \varphi_{i+1}=0$,
	\[\begin{tikzcd}
		F:\cdots & {F_{i+1}} & {F_i} & {F_{i-1}} & \cdots
		\arrow["{\varphi_{i+1}}", from=1-2, to=1-3]
		\arrow["{\varphi_i}", from=1-3, to=1-4]
		\arrow["{\varphi_{i-1}}", from=1-4, to=1-5]
		\arrow["{\varphi_{i+2}}", from=1-1, to=1-2]
	\end{tikzcd}\]
	It is useful to denote the entire chain complex as $F$ and think of it as a graded $R-$Module\footnote{Where $R$ is just trivially graded.} of the form $F=\oplus_{i \in \Z} F_i$, note that $F_n F_m \subseteq F_{n+m}$. 
	
	Note this graded module has associated with it the an endomorphism $\varphi^2=0$. We say this has degree $-1$ as it sends $a \in F_n$ to $b \in F_{n-1}$. If the grading is not relevant often this $(F,\varphi )$ is called a \textbf{differential module} with elements of $\ker \varphi $ as cycles and elements of $\mathrm{im} \varphi $ as boundaries of $F$.
	
	Furthermore the $i^{\textbf{th}}$ homology module is defined as we did previously $H_i(F)=\ker \varphi_i/ \mathrm{im} \varphi_{i+1}$ and the direct sum of all these homology modules is denoted as $H(F)=\ker \varphi/ \mathrm{im} \varphi $ and if its equal to 0 the complex is exact.
	
	
	\[ 	\cdots \to F_2	\to F_1 \xrightarrow{\varphi_1} F_0 \to 0 \]
		A complex is called a left resolution as before if its only non trivial homology is $H_0(F)=\mathrm{coker}(\varphi_1)$. If all $F_i$ projective we call it a projective resolution
		
	Dually with the arrows reversed its called an right resolution and if the modules are injective its an injective resolution.
	
	A map between differential modules $(F, \varphi), (G, \psi)$ is just $\alpha: F \to G$ such that $a \alpha = \psi \alpha $. Now if they are chain complexes we also require grading to be preserved.
	So a map between them requires a family of functions $\alpha_i: F_i \to G_i$ making the following diagrams commute,
	\[\begin{tikzcd}
		\cdots & {F_i} & {F_{i-1}} & \cdots \\
		\cdots & {G_i} & {G_{i-1}} & \cdots
		\arrow["{\varphi_i}", from=1-2, to=1-3]
		\arrow["{\alpha_i}"', from=1-2, to=2-2]
		\arrow["{\alpha_{i-1}}", from=1-3, to=2-3]
		\arrow["{\psi_i}"', from=2-2, to=2-3]
		\arrow[from=1-1, to=1-2]
		\arrow[from=2-1, to=2-2]
		\arrow[from=2-3, to=2-4]
		\arrow[from=1-3, to=1-4]
	\end{tikzcd}\]
	
	If $\alpha $ is a morphism between differential modules then it carries $\ker \varphi \to \ker \psi$ and $\image \varphi \to \image \psi$ as such it induces a map on the homology which we use the same label for $$\alpha: HF \to HG$$
	Similarly as above if grading is relevant then we have a sequence of maps,
	\[ \alpha_i: H_i F \to H_i G \]
	
	This gives rise to a natural question given two maps between complexs when is their induced homology same? This is answered by the notion of homotopy equivalence (natural transformation).
	
	\begin{definition}[Homotopy equivalence of maps]
		If $\alpha, \beta $ are maps between differential modules $(F, \varphi), (G, \psi )$ then $\alpha $ is homotopically equivalent to $\beta $ if there is a map $h: F \to G $ such that $\alpha - \beta = \psi h+ h \varphi$. If grading is relevant the picture formed is as such, we require a family of maps $h_i: F_i \to G_{i+1}$ \footnote{i.e. It has degree $1$, sometimes the subscript is dropped and just treated as $h$}
		\[\begin{tikzcd}
			\cdots && {F_i} && {F_{i-1}} && \cdots \\
			\\
			\cdots && {G_i} && {G_{i-1}} && \cdots
			\arrow["{\varphi_i}", from=1-3, to=1-5]
			\arrow["{\psi_i}"', from=3-3, to=3-5]
			\arrow["{h_{i-1}}"', from=1-5, to=3-3]
			\arrow["{\beta_i}"', shift right, from=1-3, to=3-3]
			\arrow[from=3-3, to=3-1]
			\arrow["{h_i}"', from=1-3, to=3-1]
			\arrow["{\alpha_{i-1}}", shift left, from=1-5, to=3-5]
			\arrow[from=1-1, to=1-3]
			\arrow["{h_{i-2}}"', from=1-7, to=3-5]
			\arrow[from=1-5, to=1-7]
			\arrow["{\alpha_i}", shift left, from=1-3, to=3-3]
			\arrow["{\beta_{i-1}}"', shift right, from=1-5, to=3-5]
			\arrow[from=3-5, to=3-7]
		\end{tikzcd}\]
		The intution behind this particular choice of definition is that the map $\alpha - \beta $ maps all cycles to boundaries which have zero homology. So really $\alpha- \beta$ is null homotopic, as such this relation is an equivalence relation.
	\end{definition}
	Recall the notion of singular homology. For a topological space $X$, the set of all continuous maps from $n$-simplexes to $X$ determines the singular $n$-simplex $\sigma: \Delta^n \to X$. Which in turn forms a basis that generates a free abelian group called $C_n$ whose elements are finite formal sums $
	\sum_i n_i \sigma_i$ \footnote{$n_i$ are from integers which induces a $\Z$ grading.}. This induces a boundary map $\delta_n:C_n \to C_{n-1}$ defined as \[ \delta_n(\sigma) = \sum_i (-1)^i \sigma:[v_0, \dots, \hat{v_i}, \dots, v_n]\]
	Which in turn gives rise to a chain complex $(C_\bullet, \delta_\bullet)$ with singular homology defined in the expected manner \[ H_i(X)=\ker \delta_i/ \image \delta_{i+1	} \]
	The importance of this as an example is to recall the fact that singular homology is homotopy invariant \cite[Th. ~2.10]{hatcher}. If we have continuous maps $\alpha ,\beta :X \to Y$ they induce a morphism between the singular chain complexes. And the typical notion of a homotopy between $\alpha, \beta $ as $H:X \times [0,1] \to Y $ itself induces a chain morphism $h(x)=H(x\times [0,1])$.
	
	This is often called chain homotopy. The key point that motivated the definition of homotopy equivalence is summarized in the proof of the below proposition,
	\begin{proposition}
		If $\alpha, \beta: (F,\varphi) \to (G,\psi)$ are two morphisms of differential modules and $\alpha $ is homotopically equivalent to $\beta $ then they induce the same maps on the homologies. That is to say $\alpha- \beta $ induces the zero map on homology.
	\end{proposition}
	\begin{proof}
		Let $x \in \ker \varphi $ a cycle of $F$ to show $(\alpha - \beta) (x) \in \image \psi $ i.e. in boundary of $G$. But this is clear from the definition of homotopy equivalence 
		\[ (\alpha-\beta) (x)=\psi h (x) + h \varphi(x) = \psi h(x) \]
		The proof for chain complexes with nontrivial grading is identical we just need to keep track of the indices.
	\end{proof}
	
	An important concept is that explicit modules $M, N$ can often be replaced with their projective resolutions say $F,G$ instead. And maps between modules $M \to N$ is the same as homotopy classes of maps between their resolutions $F \to G$. With the dual notion for injective resolutions (extending to cochain maps).
	
	We will see the case for projectives,
	\begin{proposition}
		Let the following chain complexes of modules be realized as follows, $M = \mathrm{coker } \varphi_1, N = \mathrm{coker } \psi _1$
		\[ F: \cdots \to F_i \xrightarrow{\varphi_i}  F_{i-1} \cdots \to F_1 \xrightarrow{\varphi_1} \ F_0\]
		\[ G: \cdots \to G_i \xrightarrow{\psi_i}  G_{i-1} \cdots \to G_1 \xrightarrow{\psi_1}  G_0  \]
		If $F$ is projective and homologies of $G$ vanish except at $H_0G=N$, then maps of modules $\beta : M \to N	$ is a map induced on $H_0$ by a map of complexes $\alpha: F \to G$ which is determined by $\beta $ upto homotopy.
	\end{proposition}
	\begin{proof}
		We have $g: G_0 \twoheadrightarrow N$ and some $f: F_0 \to M$. Consider the composition $\beta f$ which can be lifted to some map $\alpha_0 : F_0 \to G_0$ due to the surjectivity of $g$ and $F_0$ being projective	. 
		\[\begin{tikzcd}
			\cdots & {F_1} & {F_0} & M \\
			\cdots & {G_1} & {G_0} & N
			\arrow["g"', two heads, from=2-3, to=2-4]
			\arrow["\beta", from=1-4, to=2-4]
			\arrow["f", from=1-3, to=1-4]
			\arrow["{\varphi_1}", from=1-2, to=1-3]
			\arrow["{\psi_1}"', from=2-2, to=2-3]
			\arrow["{\alpha_0}", dashed, from=1-3, to=2-3]
			\arrow[from=1-1, to=1-2]
			\arrow[from=2-1, to=2-2]
		\end{tikzcd}\]
		Also this gives that $\alpha_0 \varphi_1$ sends $F_1 \to \ker g = \image \psi_1$ where the second equivalence is by exactness. So even $\alpha_0 \varphi_1 $ has a lift $\alpha_1 : F_1 \to G_1$. Continuing this inductive procedure we get the required $\alpha $ map.
		
		To show that the induced map $\alpha $ is unique consider another $\alpha' $ then consider $\alpha-\alpha'$ which would be a lifting of a zero map between $M,N$. So we wish to show $(\alpha-\alpha')_i=h_{i-1}\varphi_i +\psi_{i+1}h_i$ for some degree 1 map $h_i: F_i \to G_{i+1}$. Firstly note that the map $\alpha-\alpha'$ is a lift over the zero map. So in particular $\alpha_0 $ by the previous construction is taking $F_0 \to \image(\psi_1)$. By exactness thats equal to $\ker \psi_2$. So we have a lift $h_0: F_0 \to G_1$ such that $\psi_1 h_0=(\alpha-\alpha')_0$.
		
		Now we claim that $h_0\varphi_1 - (\alpha-\alpha')_1 $ maps into $\ker \psi_1=\image \psi_2 $ by exactness. This statement is true due to the commuting square condition of chain complexes that require $$\alpha_0 \varphi_1-\psi_1\alpha_1=0	$$ But we know $\alpha_0=\psi_1 h_0$ so we get $\psi_1(h_0\varphi_1 - \alpha_1)=0$. Now similarly due to projectiveness we can lift this to a map $h_1: F_1 \to G_2$. Continuing this gives us the required homotopy map.
		
		So $\alpha\sim \alpha'$
	\end{proof}
	
	\begin{corollary}
		Two projective resolutions $F, F'$ of some module $M$ are homotopy equivalent 
	\end{corollary}
	\begin{proof}
		Proof follows the same as second half of the previous proposition.
	\end{proof}
	\begin{corollary}
		For an additive functor $F$ and two projective resolutions $P, P'$ of some module $M$, $H_i(FP) \cong H_i(FP')$.
	\end{corollary}
	
	\subsection{Exact sequences of chain complexes}
	Consider $(A,\varphi),(B,\psi),(C,\chi)$ to be chain complexes we can define a short exact sequence of complexes as \[ 0 \to A \xrightarrow{\alpha} B \xrightarrow{\beta} C \to 0 \]
	For $\alpha, \beta $ maps of complexes as discussed above, and $\beta \alpha=0$, if for all $i$ the underlying sequence of modules is exact\[ 0 \to A_i \xrightarrow{\alpha_i } B_i \xrightarrow{\beta_i } C_i \to 0\]
	
	These maps also induce maps on the homologies $\alpha_i: H_i A \to H_i B, \beta_i: H_i B \to H_i C$. Furthermore there is a natural map \[ 	\delta_i : H_i C \to H_{i-1}A	 \] which is called the \textbf{connecting homomorphism}
	
	Before seeing how to construct this $\delta $ it is useful to have a complete picture of the data in front of us. This can be seen below,
	\[\begin{tikzcd}
		& \vdots & \vdots & \vdots \\
		0 & {A_i} & {B_i} & {C_i} & 0 \\
		0 & {A_{i-1}} & {B_{i-1}} & {C_{i-1}} & 0 \\
		0 & {A_{i-2}} & {B_{i-2}} & {C_{i-2}} & 0 \\
		& \vdots & \vdots & \vdots
		\arrow[from=2-1, to=2-2]
		\arrow["{\alpha_i}", from=2-2, to=2-3]
		\arrow["{\beta_i}", from=2-3, to=2-4]
		\arrow[from=2-4, to=2-5]
		\arrow[from=3-1, to=3-2]
		\arrow[from=4-1, to=4-2]
		\arrow["{\alpha_{i-1}}", from=3-2, to=3-3]
		\arrow["{\beta_{i-1}}", from=3-3, to=3-4]
		\arrow[from=3-4, to=3-5]
		\arrow["{\alpha_{i-2}}", from=4-2, to=4-3]
		\arrow["{\beta_{i-2}}", from=4-3, to=4-4]
		\arrow[from=4-4, to=4-5]
		\arrow["{\varphi_i}"', from=2-2, to=3-2]
		\arrow["{\varphi_{i-1}}"', from=3-2, to=4-2]
		\arrow[from=4-2, to=5-2]
		\arrow[from=1-2, to=2-2]
		\arrow[from=1-3, to=2-3]
		\arrow[from=1-4, to=2-4]
		\arrow["{\psi_i}"', from=2-3, to=3-3]
		\arrow["{\chi_i}"', from=2-4, to=3-4]
		\arrow["{\psi_{i-1}}"', from=3-3, to=4-3]
		\arrow["{\chi_{i-1}}"', from=3-4, to=4-4]
		\arrow[from=4-3, to=5-3]
		\arrow[from=4-4, to=5-4]
	\end{tikzcd}\]
	We construct via a diagram chase. Suppose $h \in H_i C= \ker \chi_i / \image \chi_{i+1} $ pick a cycle $x\in \ker \chi_i$. As $\beta_i $ is surjective we know there exists $y \in B_i$ s.t. $\beta_i(y)=x $. Now also by the fact that $x\in \ker \chi_i$ and that we have maps between chain complexes so the squares commute. We have that $\beta_{i-1}(\psi_{i}(y))=\chi_i(\beta_i(y))=\chi_i(x)=0$.
	
	Now there is some $z \in A_{i-1}$ such that $\alpha_{i-1}(z)=\psi_i(y)$ (this is due to exactness of $i-1$ sequence hence the quotient isomorphism and the above condition).
	
	As $\alpha_{i-2}$ is a monomorphism $\alpha_{i-2} \varphi_{i-1}(z)=\psi_{i-1}\alpha_{i-1}(z)=\psi_{i-1}\psi_{i}(y)=0$ so $z\in \ker \alpha_{i-1}$. Just define $\delta_i(h) $ to be the image of $z$ in $H_{i-1}A$.
	
	The above definition is well defined as it is independent of the choice of lift $x$. Pick any other lift say $x'$ now $\beta_i(x-x')=x-x=0$. So it has a preimage in $A_i$ and can be given as an embedding from $A_i \to B_i$ so $x-x' \in A_i$. $\phi_i(x-x')=\psi_i x - \psi_i x'$ which implies their images in $H_{i-1}A$ are homotopic.
	
	The fact that $\delta_i $ is a group homomorphism is simply via linearity.

	\subsection{Long exact sequences of homologies}	
	\begin{proposition}[Induced long exact sequence of homology]
		For a given short exact sequence 
		\[ 0 \to A \xrightarrow{\alpha } B \xrightarrow{\beta } C \to 0\]
		of chain complexes $(A, \varphi), (B, \psi), (C, \chi)$, then the connecting homomorphism $\delta_i: H_iC \to H_{i-1}A$ induces the following long exact sequence of homologies
			\[\begin{tikzcd}
				& \cdots & {H_iC} \\
				{H_{i-1}A} & {H_{i-1}B} & {H_{i-1}C} \\
				{H_{i-2}A} & \cdots
				\arrow[from=1-2, to=1-3]
				\arrow[from=2-1, to=2-2]
				\arrow[from=2-2, to=2-3]
				\arrow["{\delta_i}"{description}, from=1-3, to=2-1]
				\arrow["{\delta_{i-1}}"{description}, from=2-3, to=3-1]
				\arrow[from=3-1, to=3-2]
			\end{tikzcd}\]
			
			Furthermore if the chain complexes are differential modules the following triangle commutes,
			\[\begin{tikzcd}
				HA && HB \\
				& HC
				\arrow["\alpha", from=1-1, to=1-3]
				\arrow["\beta", from=1-3, to=2-2]
				\arrow["\delta", from=2-2, to=1-1]
			\end{tikzcd}\]
		
	\end{proposition}
	
	\begin{proposition}[Horseshoe lemma]
		If there is a short exact sequence of modules,
		\[ 0 \to M \to N \to P \to 0 \]
		and both $M,P$ have a projective resolutions $A, C$ 
		\[\begin{tikzcd}
			&&& 0 \\
			\cdots & {A_1} & {A_0} & M & 0 \\
			&&& N \\
			\cdots & {C_1} & {C_0} & P & 0 \\
			&&& 0
			\arrow[from=2-4, to=2-5]
			\arrow[from=1-4, to=2-4]
			\arrow[from=2-4, to=3-4]
			\arrow[from=3-4, to=4-4]
			\arrow[from=4-4, to=4-5]
			\arrow[from=4-4, to=5-4]
			\arrow[ from=4-3, to=4-4]
			\arrow[ from=2-3, to=2-4]
			\arrow[from=2-1, to=2-2]
			\arrow[from=2-2, to=2-3]
			\arrow[from=4-1, to=4-2]
			\arrow[from=4-2, to=4-3]
		\end{tikzcd}\]
		as below then $N$ also has a projective resolution $B$ which forms a short exact sequence. Also the sequence splits due to $C_i$ being projective so $B_i=A_i \oplus C_i$.
	\end{proposition}
	\begin{proof}
			First note $\epsilon_P: C_0 \to P$ lifts due to projectivity to $C_0 \to N$ also $A_0\to N$ via composition so simply define $B_0 = A_0 \oplus C_0$. This is an epi evidently via diagram chase. Also is projective as direct sum of projectives is projective. Now consider direct sum of kernel of $A_0 \to M, B_0 \to N, C_0 \to P$ and construct the direct sum again to get $F_1$.	Now we get a $3\times 3$. Exactness is due to the Snake lemma
	\end{proof}
	
	\begin{lemma}[Snake lemma]
		\[\begin{tikzcd}
			& A & B & C & 0 \\
			0 & {A'} & {B'} & {C'}
			\arrow[from=1-2, to=1-3]
			\arrow[from=1-3, to=1-4]
			\arrow[from=1-4, to=1-5]
			\arrow[from=2-1, to=2-2]
			\arrow[from=2-2, to=2-3]
			\arrow[from=2-3, to=2-4]
			\arrow["\alpha", from=1-2, to=2-2]
			\arrow["\beta", from=1-3, to=2-3]
			\arrow["\gamma", from=1-4, to=2-4]
		\end{tikzcd}\]
		The above commutative diagram induces a exact sequence \[ \ker \alpha \to \ker \beta \to \ker \gamma \to \mathrm{coker}\alpha \to \mathrm{coker}\beta \to \mathrm{coker}\gamma \]
	\end{lemma}
	\begin{proof}
		The map $\ker \gamma \to \mathrm{coker} \alpha $ is given by the connecting homomorphism.
	\end{proof}
	
	
	\textbf{TO DO: Proof of salamander and implications categorically}
	\section{Mapping cones, cylinders and double complex
	}
	
	\textbf{TODO}
	
	\section{Derived categories and Derived functors}
	
	\subsection{Derived category}
	
	\begin{definition}[Category of chain complexes]
		For arbitrary abelian category $A$
	\end{definition}
	
	\begin{definition}[Homotopy category of chain complexes]
		content
	\end{definition}
	 
	 \begin{definition}[Derived category]
	 	For an abelian category $A$ and the homotopy category $K(A)$
	 \end{definition}
	
	\subsection{Derived functor}
	Derived functors prove to be a major application of projective and injective resolutions. The motivation behind defining a derived functor is the observation that often functors fail to be exact only at an end point. It is natural to want to extend out the left/right short exact sequence in a way to get a long exact sequence.
	\begin{definition}[Left derived functor]
		Consider $F$ to be a right exact functor, if $A \in R$-Mod has a projective resolution $(P_\bullet, \varphi_\bullet)$. The $i^\textbf{th}$ left derived functor of $F$ is defined to be \[ L_iF(A)=H_i FP \] where $FP$ is the complex defined via $(FP_\bullet,F\varphi_\bullet)$
		\end{definition}
		
	The important properties of left derived functors can be characterized as follows,
	\begin{proposition}
		The left derived functor of $F$ is resolution invariant and we have
		\begin{enumerate}
			\item $L_0F=F$
			\item If $A$ is projective then all $L_iF(A)=0$.
			\item For all short exact sequence of modules \[ 0 \to A \xrightarrow{\alpha} B \xrightarrow{\beta}  C \to 0\]there is an associated long exact sequence of left derived functors
			\[\begin{tikzcd}
				&& \cdots && {L_iFC} \\
				\\
				{L_{i-1}FA} && {L_{i-1}FB} && {L_{i-1}FC} \\
				\\
				{L_{i-1}FA} && \cdots
				\arrow[from=1-3, to=1-5]
				\arrow["{\delta_i}", from=1-5, to=3-1]
				\arrow[from=3-1, to=3-3]
				\arrow[from=3-3, to=3-5]
				\arrow["{\delta_{i-1}}", from=3-5, to=5-1]
				\arrow[from=5-1, to=5-3]
			\end{tikzcd}\]
			\item The connecting homomorphisms are natural, i.e. for a map between short exact sequences 
			\[\begin{tikzcd}
				0 & A & B & C & 0 \\
				0 & {A'} & {B'} & {C'} & 0
				\arrow[from=1-1, to=1-2]
				\arrow[from=1-2, to=1-3]
				\arrow[from=1-3, to=1-4]
				\arrow[from=1-4, to=1-5]
				\arrow[from=2-1, to=2-2]
				\arrow[from=2-2, to=2-3]
				\arrow[from=2-3, to=2-4]
				\arrow[from=2-4, to=2-5]
				\arrow["\alpha"', from=1-2, to=2-2]
				\arrow["\beta"', from=1-3, to=2-3]
				\arrow["\gamma"', from=1-4, to=2-4]
			\end{tikzcd}\]
			
			the associated square of left derived functors commute,
			\[\begin{tikzcd}
				{L_iFC} & {L_{i-1}FA} \\
				{L_iFC'} & {L_{i-1}FA'}
				\arrow["{\delta_i}", from=1-1, to=1-2]
				\arrow["{L_iF\gamma}"', from=1-1, to=2-1]
				\arrow["{\delta_{i-1}}"', from=2-1, to=2-2]
				\arrow["{L_{i-1}F\alpha}", from=1-2, to=2-2]
			\end{tikzcd}\]
		\end{enumerate}
	\end{proposition}
	\begin{proof}
		For 1. with right exactness of $F$ we get  $L_0F=H_0FP=\mathrm{coker})FP_1\to FP_0)=FA$ by definition.
		
		2. As its resolution independent just take a string of left 0 projectives as the resolution.
		3. and 4. are nearly the same as seen before in the construction of connecting  homomorphism and horseshoe lemma.
	 \end{proof}
	 
	 The dual notion is as follows,
	 \begin{definition}[Right derived functor]
	 	For a left exact functor we define the right derived functor $R^iF$ as $R^iFA=H_{-1}FQ$ where $A$ is some module and $Q$ is its associated injective resolution.
	 \end{definition}
	 
	 
	\subsection{Tor functor}
	Recall that early on we saw that the tensor product over modules is right exact but fails to be left exact. So with what we have seen so far it is natural to think about the left derived functor for the tor functor $M\otimes_R -$ this is called as the Tor functor and is denoted as $\mathrm{Tor}_i^R(M,-)$. The tensor product itself is commutative up to isomorphism and the same is true for the Tor functor. Which 
	
	The name Tor comes from torsion. Recall that for some $R$-Module $M$, $x \in M$ is said to be a torsion element if there is a nonzero $r\in R $ such that $rm=0$.
	\begin{proposition}
		If $x \in R$ is not a zero divisor. Then $$Tor_1(R/x,M)=\{m \in M | xm =0\}$$
	\end{proposition}
	\begin{proof}
		content
	\end{proof}
	\begin{proposition}
		For a ring $R$ and $I,J$ ideals of $R$ we know $IJ \subseteq I \cap J$, \[ \mathrm{Tor}_1(R/I, R/J)= (I \cap J )/IJ\]
	\end{proposition}
	\begin{proof}
		content
	\end{proof}
	
	\begin{proposition}[Betti numbers of a module]
		For a local ring $R$ with maximal ideal $\mathfrak{m}$. For a $R$-Module $M$ a free resolution $(F_\bullet, \varphi_\bullet)$ of $M$ is said to be minimal if each differential $\varphi_i$ has an image contained in $\mathfrak{m}F_{i-1}$. If $F$ is minimal free resolution and rank of $F_i=b_i$ we have that \[ \mathrm{Tor}_i(R/\mathfrak{m}, M)= 	(R/\mathfrak{m})^{b_i}\] these $b_i$ are called `Betti numbers'. \footnote{Motivated by topology where the $n^{\textbf{th}}$ Betti number is the rank of the $n^{\textbf{th}}$ homology group.}
	\end{proposition}
	
	\begin{proposition}[Serre intersection](skip this)
	\end{proposition}
	
	\begin{proposition}
		For $R$-Modules $M,N$ \( \mathrm{Tor}^R(M,N) \) forms a graded associative $R$-algebra which is $Z_2$ graded commutative.
	\end{proposition}
	\begin{proposition}
		If a projective
	\end{proposition}
	\begin{proposition}[Auslander transposition]
		content
	\end{proposition}
	\subsection{Ext functor}
	Now again we know $\Hom_R(M,-)$ is left exact as it is a right adjoint. So for some module $N$ we may consider its right derived functors with respect to some injective resolution $I$ of $N$. In particular it is denoted as $R^i \Hom(M,-)N= \mathrm{Ext}_R^i(M,N):= H_{-i} \Hom(M,I)$ as seen before. This may also be computed via a \textit{projective} resolution for $M$ say $F$ as $\mathrm{Ext}_R^i (M,N)= H_{-1} \Hom(F,N)$.
	
	
	\textbf{DO EXERCISES}
	\subsubsection{Group cohomology as Ext}
	
	\subsection{Local cohomology}
	For some ring $R$ and an ideal $I$ define for an $R$-module $M$ $$\Gamma_I(M)=\{x \in M | \text{ there exists} m \in \N,\ I^n m =0\}$$
	This can be alternatively written in terms of a colimit.\[ \Gamma_I(M)= \lim_{\to_n} \Hom (R/I^n, M) \]
	
	Claim $\Gamma_I $ is a left-exact functor (\textbf{??}).
	
	So we define a right derived functor, \[ R^n \Gamma_I(M)=H_I^i(M) \] applying $\Gamma_I $ to some injective resolution of $M$.
	\subsubsection{Sheaf cohomology}
	
	\section{Spectral sequences}
	
	\section{Spectral sequence of double complex}
	
	
	
%	\subsection{Kunneth formula}
%	\subsection{Dold Kan correspondence	}
	\section{Freyd-Mitchell Embedding}
	\begin{theorem}[Freyd-Mitchell Embedding]
		If $A$ is a small abelian category there is an exact full and faithful embedding functor from $A$ into the category of modules over some ring $R$ which embeds $A$ as a full subcategory.
	\end{theorem}
	\begin{proof}
		The inclusion from $A$ to its Ind-completion is full faithful, exact, compact, preserves generators and projectives.
		
		Locally finite presentable categories preserve injective cogenerators.
		
		Cocomplete abelian category with compact projective	generator is equivalent to End(P) modules
	\end{proof}
	
	\clearpage
	\begin{appendices}
		\section{Categories}
		\subsection{Preliminary definitions}
		\begin{definition}[Category]
			A \textbf{category} consists of the following,
			\begin{itemize}
				\item Objects: A,B,C,\dots
				\item Arrows/Morphisms: f,g,h,\dots
				\item For each $ f $ there exists, $ \mathrm{dom}(f) , \mathrm{cod}(f)$ called domain and codomain of $ f $. We write $ f: A \to B $ to indicate $ A=\mathrm{dom}(f) $ and $ B=\mathrm{cod}(f) $.
				\item Given  $ f: A \to B$ and $ g: B \to C $ there exists, $ g \circ f: A \to C $ called the \textit{composite} of $ f $ and $ g $.
				\item For each $ A $, there exists $ 1_A:A\to A $ called the \textit{identity arrow} of $ A $.
				\item Arrows should also satisfy the following,
				\begin{itemize}
					\item Associativity: $ h \circ(g \circ f) = (h \circ g) \circ f,$ for all $ f:A \to B, g:B \to C, h: C \to D $.
					\item Unit: $ f\circ 1_A=f=1_B\circ f, $ for all $ f:A \to B $.	
				\end{itemize}
			\end{itemize}
		\end{definition}
		
		\begin{definition}[Initial and terminal objects]
			An object $ 0 \in \mathbf{C} $ is \textbf{initial} if for any object $ C \in \mathbf{C}, \exists!$ morphism $ 0 \to C $.\\
			An object $ 1 \in \mathbf{C} $ is \textbf{terminal} if for any object $ C \in \mathbf{C}, \exists!$ morphism $ C \to 1 $
		\end{definition}
		\begin{proposition}
			Initial and terminal objects are unique up to isomorphism.
		\end{proposition}
		For example in $\mathbf{Sets}$ the empty set is the unique initial object but a one element set is the terminal.
		
		\begin{definition}[Functor]
			For categories $\mathbf{C}, \mathbf{D}$ we define a \textbf{functor} $ F: C \to D $ to be a a mapping of objects and arrows to objects and arrows, such that
			\begin{itemize}
				\item $ F(f:A\to B) =F(f):F(A)\to F(B)$
				\item $ F(1_a)=1_{F(A)} $
				\item $ F(g \circ f) = F(g)\circ F(f)$.
			\end{itemize}
		\end{definition}
		For example the fundamental group in pointed topologies.
		\begin{definition}[Generalized elements]
			For an object $ A \in \mathbf{C} $ arbitrary arrows $ x:X\to A $ are called the \textbf{generalized elements} of $ A $ with stage of definition given by $ X $.
		\end{definition}
		
		\begin{definition}[Bifunctor]
			A \textbf{bifunctor} is any functor of two variables i.e. domain in terms of a product category.
		\end{definition}
		
		
		\begin{lemma}[Bifunctor lemma]\label{bifunctorlemma}
			A mapping $F: \mathbf{A} \times \mathbf{B} \to \mathbf{C}$ is a bifunctor if its functorial in each component and the following square commutes,
			\[\begin{tikzcd}
				{A'} & {B'} && {F(A,B)} && {F(A,B')} \\
				\\
				A & B && {F(A',B)} && {F(A',B')}
				\arrow["{F(A,g)}", from=1-4, to=1-6]
				\arrow["{F(A',g)}"', from=3-4, to=3-6]
				\arrow["{F(f,B')}", from=1-6, to=3-6]
				\arrow["{F(f,B)}"', from=1-4, to=3-4]
				\arrow["g"{description}, from=3-2, to=1-2]
				\arrow["f"{description}, from=3-1, to=1-1]
			\end{tikzcd}\]
		\end{lemma}
		
		\begin{definition}[Natural transformations]
			A \textbf{natural transformation} is a map between functors.
			
			For functors $F,G: \mathbf{C} \to \mathbf{D}$ a natural transformation $\eta: F \to G$ is a family of morphisms (in $\mathbf{D}$) which consist of \textbf{components} $\eta_X	$ which associates for every object $C \in \mathbf{C} $ a morphism between objects in $\mathbf{D}$, $\eta_C:F(C)\to G(C)$. Also components must commute naturally, in particular for $f: C \to C' $ we have $\eta_{C'} \circ F(f)=G(f)\circ\eta_C$, i.e. below diagram commutes,
			\[\begin{tikzcd}
				C && {F(C)} && {G(C)} \\
				\\
				{C'} && {F(C')} && {G(C')}
				\arrow["{\eta_C}", from=1-3, to=1-5]
				\arrow["{\eta_{C'}}", from=3-3, to=3-5]
				\arrow["{F(f)}"', from=1-3, to=3-3]
				\arrow["{G(f)}", from=1-5, to=3-5]
				\arrow["f"', from=1-1, to=3-1]
			\end{tikzcd}\]
		\end{definition}
		For example, the double dual is naturally isomorphic to the identity map in vector spaces. The Hurewicz map which states for path connected spaces there is a group homomorphism from the $n^{\textbf{th}}$ homotopy group to the $n^{\textbf{th}}$ homology group. Groups can be identified as a one object category with a morphism for each group element and now there is always a functor from $G \to X \in \mathbf{Sets}$ which forms a group action. For two such functors there is a natural transformation what we normally call equivariant maps.
		
		\begin{definition}[Diagram]
			For categories $\mathbf{J}, \mathbf{C}$ a \textbf{diagram} of type $\mathbf{J}$ in $\mathbf{C}$ is a functor $D: \mathbf{J} \to \mathbf{C}$ where $\mathbf J$ admits an indexing. This is a formalization of the notion of `diagram' we use intuitively. It can be thought of as the image of $\mathbf J$ in $\mathbf C$, the actual stucture of $\mathbf J$ is largely irrelevant.
			
			For example,
			\[\begin{tikzcd}
				& {\mathbf{J}} &&& {\text{Diagram}} \\
				\bullet && \bullet & {D_1} && {D_2}
				\arrow["f", shift left=2, from=2-1, to=2-3]
				\arrow["g"', shift right=2, from=2-1, to=2-3]
				\arrow["{D_f}", shift left=2, from=2-4, to=2-6]
				\arrow["{D_g}"', shift right=2, from=2-4, to=2-6]
			\end{tikzcd}\]
		\end{definition}
		
		\begin{definition}[Cones]
			Given $\mathbf{J}, \mathbf{C}$ and a diagram of type $\mathbf{J}$ in $\mathbf{C}$, $D: \mathbf{J} \to \mathbf{C}$ we define a \textbf{cone} to the diagram $D$ for an object (vertex) $C$ of $\mathbf{C}$ and family of arrows $c_j:C \to D_j$ for all $j \in \mathbf{J}$ such for $\alpha: i \to j$ the following commute,
			\[\begin{tikzcd}
				& C \\
				{D_i} && {D_j}
				\arrow["{c_i}"', from=1-2, to=2-1]
				\arrow["{c_j}", from=1-2, to=2-3]
				\arrow["{D_\alpha}"', from=2-1, to=2-3]
			\end{tikzcd}\]
			Furthermore we can have a morphism between cones in the natural way $\vartheta: (C, c_j) \to (C', c_j')$ making every such triangle commute,
			\[\begin{tikzcd}
				C & {C'} \\
				& {D_j}
				\arrow["\vartheta", from=1-1, to=1-2]
				\arrow["{c_j'}", from=1-2, to=2-2]
				\arrow["{c_j}"', from=1-1, to=2-2]
			\end{tikzcd}\]
			This lets us define a category of cones into $D$ denoted as $\mathbf{Cone}(D)$. Its dual is called a cocone.
			
		\end{definition}
		\begin{definition}[Limits]
			Given a diagram $D: \mathbf{J} \to \mathbf{C}$ its \textbf{limit} is a terminal object in $\mathbf{Cone}(D)$, denoted as $p_i: \lim_{\leftarrow_j}D_j \to D_i$.
			
			If $\mathbf{J}$ is finite the limit is called a finite limit.	
			
			\begin{itemize}
				\item A category has finite limits $\iff $ it has finite products and equalizers (and so pullbacks and terminal objects.)
			\end{itemize}
			
			A functor $F$ is said to \textbf{preserve limits} of type $J$ if $F(\lim_{\leftarrow} D_j) \cong \lim_\leftarrow F(D_j)$. Such a functor is called continuous.
			
			\begin{itemize}
				\item Representable functors in locally small categories are continuous.
				\item Colimts are the dual notion of limits, e.g. direct limit of groups.
			\end{itemize}
		\end{definition}
		
		
		\begin{definition}[Yoneda embedding]
			The \textbf{Yoneda embedding} is a functor $y:\mathbf{C}\to \mathbf{Sets}^{\mathbf{C}^\text{Op}}$ mapping objects to their contravariant representable functor (i.e. presheaves). In particular $y(C)=\textrm{Hom}_\mathbf{C}(-,C)$ it takes arrows to the natural transformation $yf=\mathrm{Hom}_\mathbf{C}(-,f):\mathrm{Hom}_\mathbf{C}(-,C) \to \mathrm{Hom}_\mathbf{C}(-,D)$
			
		\end{definition}
		
		\begin{proposition}[Yoneda Lemma]
			For a locally small category (i.e. small hom-sets) $\mathbf{C}$ we have $\textrm{Hom}(yC,F) \cong FC$ for $C \in \mathbf{C}$ and a functor $F \in \mathbf{Sets}^{\mathbf{C}^{op}}$. The isomorphism is natural in both $C$ and $F$.
		\end{proposition}
		\begin{proof}
			Send $\eta \in \Hom(yC, F)$ to $\eta_C(1_C) : \mathbf{C}(C,C)\to FC$	and for $ a \in FC$ consider for some $C'\in \mathbf{C}, h: C' \to C$, and $F(h)(a): \Hom(C',C) \to FC'$.	
		\end{proof}
		For locally small categories
		\begin{itemize}
			\item The Yoneda embedding $y: \mathbf{C} \to \mathbf{Sets}^{\mathbf{C}^\mathbf{op}}$ is full and faithful.
			\item $yC \cong yC' \implies C \cong C'$ for objects $C,C'$
			\item All objects in presheaves are colimits of some representable functors (in particular the end). 
			\item So Yoneda embedding is the free cocompletion of any category. In particular there is a UMP for maps from $\mathbf{C}$ to any cocomplete categories factoring through presheaves wrt Yoneda embedding
		\end{itemize}	
		An adjoint categorifies the idea of a very weak notion of equivalence sometimes even called a pseudo inverse.
		\begin{definition}[Classical definition of adjoint]
			The functors $F: \mathbf{C} \rightleftarrows \mathbf{D}: U$ form an \textbf{adjunction} between categories if there exists a natural transformation $\eta: 1_\mathbf{C} \to U \circ F$ with this unit having the following UMP,
			\[\begin{tikzcd}
				D && {U(F(C))} && {U(D)} \\
				\\
				{F(C)} && C
				\arrow["f"', from=3-3, to=1-5]
				\arrow["{\eta_C}", from=3-3, to=1-3]
				\arrow["{U(g)}", from=1-3, to=1-5]
				\arrow[dashed, from=3-1, to=1-1]
			\end{tikzcd}\]
			
			$F$ is called the left adjoint of $U$ and vice versa, and is denoted as $F \dashv U$.
		\end{definition}
		
		\begin{definition}[Hom-set definition of adjoint]
			Alternatively we also have the following formulation, $F \dashv U$ if for $C\in \mathbf{C}, D \in \mathbf{D}$ there exist natural isomorphisms $\phi: \textrm{Hom}_\mathbf{D}(FC, D) \cong \textrm{Hom}_\mathbf{C}(C, UD):\psi$
		\end{definition}
		
		\begin{definition}[Unit-counit definition]
			For $F: \mathbf{C} \rightleftarrows \mathbf{D}:U$ and natural transformations $\eta: 1_\mathbf{C} \to U \circ F$ and $\epsilon: F \circ U \to 1_\mathbf{D}$ namely unit and counit. 
			
			We say $F \dashv U$ iff the following triangle identities hold,
			\[\begin{tikzcd}
				UD && UD \\
				& UFUD \\
				FC && FC \\
				& FUFC
				\arrow["{1_{UD}}", from=1-1, to=1-3]
				\arrow["{\eta_{UD}}"', from=1-1, to=2-2]
				\arrow["{U_{\epsilon_D}}"', from=2-2, to=1-3]
				\arrow["{1_{FC}}", from=3-1, to=3-3]
				\arrow["{F_{\eta_C}}"', from=3-1, to=4-2]
				\arrow["{\epsilon_{FC}}"', from=4-2, to=3-3]
			\end{tikzcd}\]
		\end{definition}
		The unit-counit characterization is useful theoretically but we typically only use the Hom-set characterization for all practical purposes.
		
		\subsubsection{Examples of adjoints}
		Free-forgetful adjunction.Typically the forgetful functor usually has a left adjoint which is vaguely `free'. Obviously neither `free'-ness or `forgetful`-ness have concrete categorical definitions but differ from case to case. We will use it for proving injectives are enough.
		
		Tensor-Internal Hom adjunction, which we shall see shortly has a special case over topologies where it is called the smash functor hom adjunction. A special case of this (over $S^1$) is the suspension-loop adjunction which yields an example of the Eckmann-Hilton duality. 
		
		For arbitrary diagonal functor over some diagram, its right adjoint is limits and left adjoint is its colilmit.
		
		\subsection{Monoidal categories}
		A \textbf{monoidal category} is a category $\mathbf{C}$ with a bifunctor $ \otimes: \mathbf{C} \times \mathbf{C} \to \mathbf{C}$, a `unit' element $I$, and natural isomorphisms that make the functor associative and unital with $I$ as expected. It is a generalization of the notion of a `tensor product'. Its used to define enriched categories.
		
		This can be formalized as follows, there exists $I \in \mathbf{C}$ and natural isomorphisms,
		\begin{itemize}
			\item $\alpha_{ABC}: A \otimes (B \otimes C) \to (A \otimes B) \otimes C$
			\item $\lambda_A : I \otimes A \to A$ (read as left)
			\item $\rho_A: A \otimes I \to A$ (read as right)
		\end{itemize}
		
		And the following diagrams commute,
		\[\begin{tikzcd}
			& {(A\otimes B)\otimes (C \otimes D)} \\
			{A \otimes(B \otimes(C\otimes D))} && {((A\otimes B)\otimes C)\otimes D} \\
			{A \otimes ((B \otimes C) \otimes D)} && {(A \otimes (B\otimes C))\otimes D}
			\arrow[from=2-1, to=1-2]
			\arrow[from=1-2, to=2-3]
			\arrow[from=3-1, to=3-3]
			\arrow[from=3-3, to=2-3]
			\arrow[from=2-1, to=3-1]
		\end{tikzcd}\]
		\[\begin{tikzcd}
			{A\otimes(I\otimes B)} && {(A\otimes I)\otimes B} \\
			& {A\otimes B}
			\arrow[from=1-1, to=1-3]
			\arrow[from=1-3, to=2-2]
			\arrow[from=1-1, to=2-2]
		\end{tikzcd}\]
		
		These coherence conditions were first introduced by Maclane and a textbook presentation for it is there in his book \cite[Sec. ~VII Monoids]{lane1998categories}. The informal reasoning behind the coherence conditions is that all diagrams made with the given natural isomorphisms commute.
		
		We have a specific notion of a \textbf{symmetric monoidal category} wherein there exists another natural isomorphism called \textbf{braiding} which just `switches' the order of the tensor product, i.e. $B_{A,B}: A \otimes B \to B \otimes A  $ furthermore $B_{A,B}B_{B,A}=1_{A \otimes B}$, i.e. the tensor product is commutative!
		
		The coherence conditions for this includes two hexagon conditions which essentially say that braiding is associative across three objects as expected. This is excluded for brevity but can be seen in detail in \cite{nlab:braided_monoidal_category}	
		
		\subsection{External Hom and internal hom functor}
		The usual notion of the Hom functor involves simply considering the Hom-sets between two objects in a locally small category as a set. This is sometimes called the \textbf{external} Hom functor to differentiate it from the internal hom.
		\begin{definition}[Covariant Hom-functor]
			$\Hom(X,-): \mathbf{C} \to \mathbf{Sets}$ sending each object $Y \in \mathbf{C}$ to the set of morphisms between $X,Y$ and mapping morphisms naturally as composition, i.e., $\Hom(X,-)$ sending $f:Y \to Z$ to $\Hom(X,f): \Hom(X,Y)\to \Hom(X,Z)$
			sending $g\in \Hom(X,Y)$ to $g \circ f$.
		\end{definition}
		
		Perhaps much more importantly however, $\Hom(-,-)$ can be considered as a bifunctor in the natural way from $\mathbf{C}^{\mathrm{op}}\times \mathbf{C} \to \mathbf{Sets}$, see \cite{nlab:hom-functor}.
		
		The notion of an \textbf{internal hom} is essentially a functor that behaves the same as a Hom functor but takes values into the category itself instead of into $\mathbf{Sets}$.
		Internal homs are well defined for categories which are symmetric monoidal category as follows.
		
		\begin{definition}[Internal hom]
			For a symmetric monoidal category ($\mathbf{C}, \otimes$), the internal hom is a functor \[ [X,-]: \mathbf{C}^{\mathrm{op}} \times \mathbf{C} \to \mathbf{C}\]
			characterized by a pair of adjoint functors $X \in \mathbf{C}$ \[ -\otimes X \dashv [X,-]: \mathbf{C}  \to \mathbf{C}\] If these exist it is called a symmetric closed monoidal category
		\end{definition}
		
		\begin{proposition}[Internal hom bifunctor]\label{hombifunctor}
			For a symmetric monoidal category ($\mathbf{C}, \otimes$), the internal hom is a unique bifunctor \[ [-,-]: \mathbf{C}^{\mathrm{op}} \times \mathbf{C} \to \mathbf{C}\]
			corresponds to the previous definition, and also \[  \Hom (A \otimes B, C) \cong \Hom[A, [B,C]]\] where this is the external hom
		\end{proposition}
		\begin{proof}
			The natural isomorphisms exist due to the homset definition of adjunction keeping the components fixed at $B$ and $B, C $. Furthermore bifunctorality is a consequence of bifunctor lemma Prop. \ref{bifunctorlemma}.
		\end{proof}
		
		
		\begin{theorem}[Tensor-Internal Hom adjunction]\label{tensorinternalhom}
			\[ [A\otimes B, C] \cong [A, [B,C]] \]
		\end{theorem}
		\begin{proof}
			For some $A,B,C,X \in \mathbf{C}$ apply first the maps from the above proposition 
			\begin{align*}
				\Hom(X,[A\otimes B, C]) &\cong \Hom (X \otimes A \otimes B, C)
				 \cong \Hom((X \otimes A)\otimes B, C) \\&\cong \Hom(X \otimes A, [B,C]) \cong \Hom(X,[A,[B,C]])
			\end{align*}
			Now via Yoneda we get $[A \otimes B, C] \cong [A, [B,C]]$.
		\end{proof}
		
		\begin{lemma}[Recovery of external hom from internal hom]
			\[ \Hom[A,B]\cong \Hom(I,[A,B]) \]
		\end{lemma}
		
		\subsubsection{Examples of internal homs}
		In general internal homs and external homs need not correspond, intuitively the internal hom contains a lot more information than just the external hom. And for monoidal unit $I$ maps from $I \to [A,B]$ correspond to the external homs $A \to B$.
		
		Heyting algebras correspond to intuitionistic propositional logic. And its internal hom corresponds to its preorder that is to say $p \leq q $ iff $  p \implies q$. This is quite interesting as external homs in Heyting algebras are simply sets with at most one element while internal homs are very illuminating. Furthermore extending this analogy to type theory the idea of the tensor internal hom adjunction is just the notion of `currying'.
		
		Example In the category of chain complexes and $I$ being the monoidal unit as before. Maps from $I \to A$ for $A$ a \textbf{(COMPLETE THIS)}
		
		\subsection{Enriched categories}
		Very often the Hom-set of a category instead of just being a set may have additional structure (or we may want it to have additional structure). In order to formalize this notion we replace Hom-sets with Hom-\textit{objects} which themselves are objects in another category say $K$. We say its enriched over $K$. This choice of $K$ is typically taken to be monoidal in order to define composition of Hom-sets via the monoidal product \footnote{This monoidal property can further be dropped and replaced with bicategories (i.e. a 2-category without strict associativity) \cite{garner2015enriched}. But this is out of the scope of this article.}. A category $\mathbf{C}$ is said to be enriched over some monoidal category $K$ if ordered pairs in $\mathbf{C}$ can be represented as objects in $K$ with composition defined via the monoidal product. The coherence conditions required to define this are excluded for brevity but can be seen in \cite[Chp. ~3]{riehl_2014} which presents it without proof.
		
		The actual definition isn't important here, we only mention it as the main object of study (abelian categories) is a special case of an enrichment over category of abelian groups.
		
		For example categories enriched over the category of finite categories are just 2-categories!
		\subsection{Abelian Categories}
		There is a chain of conditions regarding `abelian'-ness of categories which is roughly understood as follows,
		\[ \textbf{Abelian} \subseteq \textbf{Pre-Abelian} \subseteq \textbf{Additive} \subseteq \textbf{Ab-Enriched}\]
		The motivation behind them is to have categories which resemble algebras.
		
		Ab-Enriched categories are categories such that for objects $A,B \in \mathbf{C}$ the external hom set $\Hom(A,B)$ has the structure of an abelian group, furthermore it has a well defined notion of composition (which is bilinear due to the monoidal product in Ab), $\Hom(A,B)\otimes \Hom(B,C) =\Hom(A,C)$. 
		\begin{proposition}
			In Ab-Enriched categories intial and terminal objects coincide (it is often called the zero object)
		\end{proposition}
		\begin{proof}
			Let $\mathbf{C}$ be an Ab-Enriched category. Note that the Hom-sets between objects have `zero morphisms', i.e. arrows in the Hom-set which behave like the additive identity in the Ab group induced by it. In particular for $0_{A,B}\in \Hom(A,B)$ we have the property that if $f:B \to C$ then $f\circ 0_{A,B}=0_{A,C}$ and $g: A \to D$ then $0_{A,B}\circ g=0_{D,B}$.
			
			Now suppose $0 \in \mathbf{C}$ is initial so there is a unique morphism $0\to 0$ so in its Hom-set its both the additive inverse and the identity. So for any $f:X \to 0$ we can say that by the zero morphism property $f=0$ so also $0$ is terminal.
		\end{proof}
		\begin{proposition}
			In Ab-Enriched categories finite coproducts coincide with finite products (i.e. biproducts) \footnote{This also holds over categories enriched over commutative monoids.}
		\end{proposition}
		\begin{proof}	
			Let $\mathbf{C}$ be an Ab-enriched category and $A,B\in \mathbf{C}$ consider the product $A\times B$, which is determined by the following UMP,
			\[\begin{tikzcd}
				& X \\
				A & A\times B & B
				\arrow["{p_1}"', from=2-2, to=2-1]
				\arrow["{p_2}", from=2-2, to=2-3]
				\arrow["u", dashed, from=1-2, to=2-2]
				\arrow["{x_1}"', from=1-2, to=2-1]
				\arrow["{x_2}", from=1-2, to=2-3]
			\end{tikzcd}\]
			Consider $A$ and $B$ in place of $X $ in the diagram. By the UMP we have $q_1: A \to A\times B, q_2: B \to A\times B$
			\[\begin{tikzcd}
				A && B \\
				& {A\times B} \\
				A && B
				\arrow["{p_1}"', from=2-2, to=3-1]
				\arrow["{p_2}", from=2-2, to=3-3]
				\arrow["{q_2}"', from=1-3, to=2-2]
				\arrow["{q_1}"', from=1-1, to=2-2]
				\arrow["{1_A}"', from=1-1, to=3-1]
				\arrow["{1_B}", from=1-3, to=3-3]
			\end{tikzcd}\]
			So $p_1q_1=1_A$ and $p_2q_2=1_B$ also $p_1q_2=p_2q_1=0$.
			
			Now note that $q_1p_1+q_2p_2=1_{A\times B}$ as $p_1(q_1p_1+q_2p_2)=p_1$ and $p_2(q_1p_1+q_2p_2)=p_2$. Claim this $q_1,q_2$ determine a coproduct $A +B$.
			
			We wish to show the following UMP holds for some arbitrary $C \in \mathbf{C}$
			\[\begin{tikzcd}
				A & C & B \\
				& {A\times B} \\
				A && B
				\arrow["{q_1}"', from=1-1, to=2-2]
				\arrow["{q_2}", from=1-3, to=2-2]
				\arrow["{p_1}"', from=2-2, to=3-1]
				\arrow["{p_2}", from=2-2, to=3-3]
				\arrow["{1_B}"', from=1-3, to=3-3]
				\arrow["{1_A}"', from=1-1, to=3-1]
				\arrow["{r_1}", from=1-1, to=1-2]
				\arrow["{r_2}"', from=1-3, to=1-2]
				\arrow["{f}"',dashed, from=2-2, to=1-2]
			\end{tikzcd}\]
			Define $f: A\times B \to C$ as $f=r_1p_1+r_2p_2$. Now $fq_1=r_1$ and $fq_2=r_2$ if we show uniqueness of $f$ we are done.
			
			Say $f'$ then $(f-f')1_{A \times B}=(f-f')(q_1p_1+q_2p_2)=0$. So $f=f'$.
			
			
		\end{proof}
		
		
		\begin{definition}[Additive category]
			An Ab-Enriched category which has all finite coproducts.
		\end{definition}
		
		Functors between additive categories are called \textit{additive functors}. And can be realized as functors which preserve additivity of homomorphisms between modules, $F(f+g)=F(f)+F(g).$
		
		Before proceeding further it is important to think about kernels and cokernels in the categorical sense.
		\begin{definition}[Kernel]
			A kernel is a pullback of a morphism $f:A \to B$ and the unique morphism from $0 \to B$. Provided initials and pullbacks exist.
			
			\[\begin{tikzcd}
				{\ker f} && 0 \\
				\\
				A && B
				\arrow["f", from=3-1, to=3-3]
				\arrow[from=1-3, to=3-3]
				\arrow[from=1-1, to=1-3]
				\arrow[from=1-1, to=3-1]
			\end{tikzcd}\]
		\end{definition}
		The intuition behind this definition is that alternatively it is seen as an equalizer of a function $f:A \to B$ and the unique zero morphism $0_{A,B}$. The kernel object is the part of the domain that is 'going to zero'. \footnote{A minor point to note is that in the case of Ab-Enrichments the `zero' in the Hom-sets isn't a terminal, its Hom-set specific. When you assume a Ab-Enriched category has a initial 0 however this matches up with our intuition.}
		
		
		\begin{definition}[Pre-abelian categories]
			An additive category with all morphism having kernels and cokernels.
		\end{definition}
		The above definition is equivalent to saying a pre-abelian category is a Ab-Enriched category with all finite limits and colimits. This is a consequence to the fact that categories have finite limits iff it has finite products and equalizers \cite[Prop.~5.21]{Awodey}. And we know equalizers exist because equalizers of two morphisms is just the kernel of $f-g$.
		
		
		
		
		\begin{definition}[Abelian category]
			Pre-additive categories for which each mono is a kernel and each epic is a cokernel.
		\end{definition}
		Largely the purpose of abelian categories were motivated by wanting to generalize homological methods and to unify various (co)homology theories. It was defined in the modern formulation by Grothendieck in his Tohuku paper \cite{grothendieck1957quelques}. We never directly reference this paper for its mathematical content but it is interesting from a historical perspective.
		\subsubsection{Examples}
		Some examples of abelian categories are as follows,
		\begin{enumerate}
			\item \textbf{The category of modules.}
			\item\textbf{ A very unique example is that the category of representations of a group is abelian!} A group representation is a group action of a group $G$ on some vector space $V$ via invertable maps, alternatively just a group homomorphism $G\to GL(V)=\mathrm{Aut}(V)$. When posed categorically it becomes a littly silly. This is nothing but a functor between `abelian' objects. The category of representations really is just all such representations between $G$ and all automorphisms of vector spaces.
			
			To show the category of representations is abelian (i.e. the functor category). Note the morphisms in between representations $G \to \mathrm{Aut}(V)$ themselves form a vector space. Direct sums are also easy to define naturally. The only difficult notion is to show all monos are some kernel and all epics are some cokernel. Consider some representations $f: A \to \mathrm{Aut}(V), g: A \to \mathrm{Aut}(W)$. If we assume there is a natural transformations $\alpha: f \to g$. It is monic iff its kernel is trivial. (Left cancellation between composition of function with inclusion and zero map.)	
			\item \textbf{Category of sheaves of abelian groups on some topological space.}
			
			To recall the definition of a sheaf. The first example we naturally see is the sheaf of continuous real valued scalar functions of $n$ variables or $n$ times differentiable functions on some open $ U$ set of $\R^n$. The following properties form useful motivations,
			\begin{enumerate}
				\item The original set of functions have restrictions down to any other open $V \subset V$, namesly $f \mapsto f|_V$.
				\item 
			\end{enumerate}
			This gives sufficient motivation for the typical definition of a pre-sheaf and a sheaf,
			\begin{definition}[Presheaf]
				For a category $C$ a presheaf is any functor $F: \mathbf{C}^{\mathrm{op}}\to \mathbf{Sets}$.
			\end{definition}
			In particular in the case for a topological space $X$ a presheaf of groups (or any algebraic object) on $X$ (in truth the set of the lattice of open sets of $X$ ordered by inclusion) is a some contravariant functor $F$ which sends open sets $U \subseteq X$ to some $F(U)$, it respects inclusions (i.e. there for open sets $V \subseteq V $ is a natural transformation $\rho_{UV}: F(U)\to F(V)$ in the form of a restriction). Furthermore, function composition, unitals and empty sets going to empty sets hold (to make it a category). Note that all these notions of presheaves are really just a special case of the categorical definition where the sheaf of groups is really just a group object in the categorical presheaf.
			\begin{definition}[Sheaf of sets on a topology]
				A sheaf of a topology $X$ is a presheaf which satisfies two additional properties, for open sets $U \in X$ and open covers ${U_i}$ of $U$
				\begin{enumerate}
					\item (Locality)  \textbf{A section}, i.e. an element $s \in F(U)$ goes to zero restricted at $U_i$ for all $i$ implies $s=0$.
					\item (Gluing) If there is a collection of sections $s_i \in F(U_i)$ such that $s_i|_{U_i \cap U_j}=s_j|_{U_i \cap U_j}$ for all $i,j$ then there is some $s \in F(U)$ such that $s|_{U_i}=s_i$ for all $i$.
				\end{enumerate}
				
				These two conditions can be written is short as just saying we require $F(U)$ to be the equalizer for the following diagram
				\[\begin{tikzcd}
					{\prod_{i \in I} F(U_i)} && {\prod_{i,j}F(U_i \cap U_j)}
					\arrow[shift right=2, from=1-1, to=1-3]
					\arrow[shift left=2, from=1-1, to=1-3]
				\end{tikzcd}\]
			\end{definition}
			
			Now finally we get back to the original example. The category of sheaves of abelian groups on a topological space form a abelian category. Additivity is natural due to the functorial nature of $F$. A slightly unsatisfying proof is due to `sheafification', i.e. the left adjoint to the inclusion functor from sheaves into presheaves. Presheves of abelian groups can be understood to have all the required properties to be an Abelian category due the functorial representation. Now due to the following result \cite{stacks1} we can extend this notion to the sheaves via sheafification.
		\end{enumerate}
		\subsubsection{Important results}
		There are a few concepts and definitions relevant in the conversation of abelian categories which we will list out here for completeness. Firstly is the notion of \textbf{exact functors} the typical notion of a functor carrying forward exact sequences. With the prefix of left/right added to determine it carrying forward only left or right sides of the exact sequence.
		
		
		\begin{proposition}\label{adjointinjective}
			Given a pair of adjoint functors $F \dashv U$ between abelian categories $F:\mathbf{C} \rightleftarrows \mathbf{D}:U$ if the left adjoint $F$ is exact, faithful and if $ \mathbf{D}$ has enough injectives also $\mathbf{C}$ has enough injectives.
		\end{proposition}
		\begin{proof}
			content...
		\end{proof}
	\end{appendices}
	\appendix
	
	

	
	\bibliographystyle{apalike}
	\bibliography{references}
	
	
\end{document}
